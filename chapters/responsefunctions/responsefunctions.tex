\chapter{Computing Response Functions \& Polaron Mobility}
\label{chap:fifth}

\thesisepisrcyear{``This might suggest that the so-called imaginary time is really the real time, and that what we call real time is just a figment of our imaginations\ldots\ a scientific theory is just a mathematical model we make to describe our observations: it exists only in our minds. So it is meaningless to ask: which is real, ``real'' or ``imaginary'' time? It is simply a matter of which is the more useful description.''}{Stephen Hawking}{A Brief History of Time}{1988}

\chapterintrobox{This is the introduction paragraph.}

\section{Numerical Evaluation of the Memory Function} 

Previous work~\cite{feynman_mobility_1962, devreese_optical_1972}, including my own (see Appendices), made use of the `doubly-oscillatory' contour-rotated integral for the complex memory function in Eq.~(\ref{eqn:fhip_chi}). The imaginary component of the memory function is (Eqs.~(47) in Ref.~\cite{feynman_mobility_1962}),
\begin{equation}\label{eqn:powerseriesimag}
        \Im{\chi(\Omega)} = \frac{2\alpha \omega_0^2}{3\sqrt{\pi}}
        \frac{(\hbar\omega_0\beta)^{\frac{3}{2}} \sinh(\hbar \Omega \beta
        / 2)}{\sinh(\hbar \omega_0 \beta / 2)} \left(\frac{v}{w}\right)^3
        \int_0^\infty d\tau \frac{\cos(v \omega_0 \tau) \cos(\omega_0
        \tau)}{\left[ \omega_0^2 \tau^2 + a^2 - b \cos(v \omega_0 \tau)
        \right]^{\frac{3}{2}}},
\end{equation}
where $a^2 \equiv\left(\hbar\omega_0\beta/2\right)^2 + R \hbar\beta\omega_0 \coth(\hbar \beta \omega_0 v/2)$, $b \equiv R \hbar\beta\omega_0 / \sinh(\hbar\beta\omega_0 v / 2)$ and $R \equiv (v^2 - w^2) / (w^2 v)$, and where $\tau$ labels \textit{imaginary time} compared to $t$ that labels \textit{real} time in Eqs. (\ref{eqn:fhip_chi}). Additionally, the contour integral for the real component of the memory function, derived by us (see Appendix A), is,
\begin{equation}\label{eqn:powerseriesreal}
    \begin{aligned}
        \Re{\chi(\Omega)} &= \frac{2\alpha\omega_0^2}{3\sqrt{\pi}}
        \frac{(\hbar\omega_0\beta)^{\frac{3}{2}}}{\sinh(\hbar\omega_0\beta/2)}\left(\frac{v}{w}\right)^3\\
        &\times \Biggl\{\sinh\left(\frac{\hbar \Omega
        \beta}{2}\right)\int_0^\infty d\tau \frac{\sin(\Omega \tau)
        \cos(\omega_0 \tau)}{\left[\omega_0\tau^2 + a^2 - b\cos(v \omega_0
        \tau)\right]^{\frac{3}{2}}} \\
        &\quad- \int^{\frac{\hbar\beta}{2}}_0 d\tau
        \frac{(1-\cosh(\Omega(\tau - \hbar\beta/2))
        \cosh(\omega_0\tau)}{\left[a^2 - \omega_0^2 \tau^2
        - b\cosh\left(v\omega_0\tau\right) \right]^{\frac{3}{2}}} \Biggr\}.
    \end{aligned}
\end{equation}
The imaginary component of the memory function can be expanded in Bessel functions (originally, the derivation was outlined in Refs.~\cite{feynman_mobility_1962, devreese_optical_1972}. Still, in Appendix B, we provide an in-depth derivation) and the real component in terms of Bessel and Struve functions (see Appendix C for a derivation of the expansion that we believe to be new).

However, we found that the cost of evaluating these expansions became large at low temperatures, requiring arbitrary precision numerics to reach converged solutions slowly. Devreese et al.~\cite{devreese_optical_1972} found an alternative analytic expansion for the real component but similarly found it to have poor convergence for all temperatures, opting to transform the integrand to one with better convergence. 

Instead of using any of the contour integrals or power-series expansions, we found that directly numerically integrating Eq.~(\ref{eqn:memoryfunction}) using an adaptive Gauss-Kronrod quadrature algorithm leads to faster convergence and controlled errors. Asymptotic limits of these contour integral expansions, especially at low temperatures, may still prove helpful.

\section{Violation of the Mott-Ioffe-Regel criterion versus Planckian bound}

The usual MIR criterion puts bounds on transport coefficients of the Boltzmann equations for quasiparticle-mediated transport, where localised wavepackets are formed from superpositions of single-particle Bloch states. Beyond these bounds, the mean free path of a quasiparticle is of order or smaller than its Compton wavelength, where it is no longer possible to form a coherent quasiparticle from superpositions of Bloch states due to the uncertainty in the single-particle state positions.

Violation of the MIR limit is commonly observed in strongly correlated systems at high temperatures. It is often used to suggest that quasiparticle physics does not describe transport in these materials. The ``thermal'' MIR criterion also depends on the validity of the Boltzmann description. Still, it is subtly different to the usual MIR criterion as clearly explained by Hartnoll and Mackenzie~\cite{mousatov_planckian_2020, hartnoll_colloquium_2022} who refer to it instead as a ``Planckian bound''. Whereas the MIR criterion discerns the ability to form coherent particles from the superposition of Bloch states, the Planckian bound describes the ability of quasiparticles to survive inelastic many-body scattering. 

Despite this, the Feynman variational method, a quasiparticle theory, predicts mobilities outside the Planckian bound, which agrees with diagMC mobility predictions, as shown in the next section.

This cautions against using semi-classical mobility theories, such as Bloch waves, as their charge-carrier wavefunction ansatz to model polar materials. 

\section{Comparison to Diagrammatic Monte Carlo}

\begin{figure}[t]
    \centering
    \includegraphics[width=.49\textwidth]{figures/frohlich-3d-mobility-alpha-2.5-temp-00625to32-COLOUR.pdf}
    \includegraphics[width=.49\textwidth]{figures/frohlich-3d-mobility-alpha-4-temp-00625to32-COLOUR.pdf}
    \includegraphics[width=.49\textwidth]{figures/frohlich-3d-mobility-alpha-6-temp-00625to32-COLOUR.pdf}
    \caption{Temperature dependence of the polaron mobility. The red dotted lines give the Mott-Ioffe-Regel (MIR) threshold. Left: Mobility obtained from the thermal FHIP theory with $v$ and $w$ variational parameters calculated for each temperature, for Fr\"ohlich alphas ranging from $\alpha = 1$ to  $12$. The temperature is given in units of the phonon frequency $\omega$. Right: Mobility obtained by~\cite{mishchenko_polaron_2019} using the diagrammatic Monte Carlo method for $\alpha = 6$. The temperature is in units of the phonon frequency $\Omega$.}
    \label{fig:mishchenko2}
\end{figure}

\begin{figure}[t]
    \centering
    \includegraphics[width=.49\textwidth]{figures/Mischenko_comparison.pdf}
    \includegraphics[width=.49\textwidth]{figures/medium.png}
    
    \caption{Polaron mobility for $\alpha = 6$. Left: Mobility obtained from the thermal FHIP theory with $v$ and $w$ variational parameters calculated for each temperature. The electric field frequency and temperature are given in units of the phonon frequency $\omega$. Right: Mobility obtained by~\cite{mishchenko_polaron_2019} using the diagrammatic Monte Carlo method. The electric field frequency and temperature are in units of the phonon frequency $\Omega$.}
    \label{fig:mishchenko}
\end{figure}

Mishchenko et al.~\cite{mishchenko_polaron_2019} recently used diagrammatic Monte Carlo (diagMC) calculations to investigate the violation of the so-called ``thermal'' analogue to the Mott-Ioffe-Regel (MIR) criterion in the Fr\"ohlich polaron model. This ``thermal'' MIR criterion is perhaps better called the \emph{Planckian bound}~\cite{hartnoll_colloquium_2022} under which a quasiparticle is stable to inelastic scattering. 
% A quasiparticle can form from the superposition of single-particle Bloch states. 
% If the quasiparticle undergoes elastic scattering, these single-particle states are well-defined, long-lived energy eigenstates. 
% If inelastic scattering is introduced, energy can be transferred between the single-particle states to spread the single-particle energy. 
For the quasiparticle to propagate coherently, the inelastic scattering time $\tau_{\text{inel}}$ must be greater than the ``Planckian time'' $\tau_{\text{Pl}} = \hbar / k_B T$. For the polaron mobility $\mu$, this requires $\mu \gtrsim \frac{e\hbar}{M k_B T}$. This Planckian bound can be reformulated into Mishchenko's~\cite{mishchenko_polaron_2019} ``thermal'' MIR criterion for the validity of the Boltzmann kinetic equation, $l >> \lambda$ where $l$ is the mean free path, and $\lambda$ the de Broglie wavelength, of the charge carrier.

The anti-adiabatic limit ($k_BT \ll \hbar\omega_0$) corresponds to the weak-coupling limit ($\alpha \ll 1$), where the perturbative theory results for the mobility is (Eq.~(5) in~\cite{mishchenko_polaron_2019})
\begin{equation}\label{eqn:coldmobility}
    \begin{aligned}
        \mu &= \frac{e}{2M\alpha\omega_0} e^{\hbar \omega_0/k_BT} \\ 
        &= \frac{e}{2m^*\omega_0} \left(\frac{1}{\alpha} - \frac{1}{6}\right) e^{\hbar \omega_0/k_BT}, \quad (k_BT \ll \hbar\omega_0, \alpha \ll 1),
    \end{aligned}
\end{equation}
where $M = m_b / (1 - \alpha / 6)$ is the effective mass renormalisation of the polaron. In the adiabatic regime ($k_B T \gg \hbar\omega_0$), the mobility is obtained from the kinetic equation as (Eq.~(6) in~\cite{mishchenko_polaron_2019})
\begin{equation}\label{eqn:hotmobility}
    \mu = \frac{4 e \sqrt{\hbar}}{3\sqrt{\pi}\alpha M \sqrt{\omega_0 k_B T}}, \quad (k_BT \gg \hbar\omega_0),
\end{equation}
which is valid even when $\alpha$ is not small.

Fig.~\ref{fig:mishchenko_fig2} is a comparison with Fig.~2 in~\cite{mishchenko_polaron_2019} of the polaron mobility at $\alpha = 2.5$. At low temperatures ($k_BT \lesssim \hbar\omega_0 / 2$), the exponential behaviour matches the low-temperature mobility in Eq.~(\ref{eqn:coldmobility}). As in~\cite{mishchenko_polaron_2019}, there appears to be a delay in the onset of the exponential behaviour for $k_BT < \hbar\omega_0$. Likewise, the MIR criterion is violated over the temperature range $0.2 < k_BT/\hbar\omega_0 < 10$. At high temperatures, the FHIP mobility (Eq.~(\ref{eqn:freq_dep_mobility})) has the same $1/\sqrt{T}$ dependence as Eq.~(\ref{eqn:hotmobility}).

In Fig.~\ref{fig:mishchenko_fig3}, I compare the temperature dependence of the FHIP polaron mobility with the diagMC polaron mobility (Fig. 3~\cite{mishchenko_polaron_2019}) at $\alpha = 6$. The diagMC polaron mobility exhibits non-monotonic behaviour at $\alpha = 6$, with a clear local minimum around $k_BT = \hbar\omega_0$. Here, we see similar non-monotonic behaviour in the FHIP mobility with a small local minimum around $k_B T = \hbar\omega_0$ too. However, compared to the diagMC mobility, the local minimum of the FHIP mobility is shallower. The onset of this minimum in the FHIP mobility begins around $\alpha = 6$, with the minimum deepening at stronger couplings ($\alpha = 8\ \&\ 10$). Like the diagMC mobility, the high-temperature limit is recovered after a maximum at $ k_B T /\hbar\omega_0 \sim \alpha$, which shifts with larger $\alpha$. The minimum too appears to be $\alpha$-dependent, occurring at $k_BT/\hbar\omega_0 \sim 1$ for $\alpha = 6$ or $k_BT/\hbar\omega_0 \sim 1.5$ for $\alpha = 10$.

In Fig.~\ref{fig:mishchenko_fig4}, I compare the temperature and frequency dependence of the FHIP polaron mobility with the diagMC polaron mobility (Fig. 4 in~\cite{mishchenko_polaron_2019}) at $\alpha = 6$ for temperatures $T = 0.5\omega_0, 1.0\omega_0, 2.0\omega_0$. The FHIP mobility, obtained by integrating Eq.~(\ref{eqn:freq_dep_mobility}), has similar temperature dependence to the diagMC mobility but differs in frequency response. 

The FHIP mobility shows extra peaks where the first peak is blue-shifted compared to the diagMCs single peak. In~\cite{devreese_optical_1972, de_filippis_validity_2006}, it is shown that these extra FHIP mobility peaks correspond to the polaron quasiparticle's internal relaxed excited states. These internal states correspond to multiple phonon scattering processes. For $k_B T/\hbar\omega_0 = 0.5$, the first peak around $\Omega/\omega_0 \sim 6$ corresponds to one-phonon processes, the peak at $\Omega / \omega_0 \sim 10$ corresponds to two-phonon processes, and so on. This is more clearly seen by analysing the memory function $\chi(\Omega)$ (Eq.~(\ref{eqn:fhip_chi})) at zero temperature, which similarly has peaks at $\Omega / \omega_0 = 1 + nv$, where $n = 0, 1, 2, ...$ and $v$ is one of the Feynman variational parameters (c.f. Fig.~\ref{fig:memory_function}). These peaks in the memory function correspond to the same Frank-Condon states. As the temperature increases, the first few peaks become more prominent and broaden due to an increased effective electron-phonon interaction. Eventually, the excitations can no longer be resolved at high temperatures. 

The Feynman variational model of the electron harmonically coupled to a fictitious massive particle (c.f. Section \ref{section:spectra}) lacks a dissipative mechanism for the polaron such that the polaron state described by this model does not lose energy and has an infinite lifetime. However, in de Filippis et al.~\cite{de_filippis_validity_2006}, dissipation is included in this model at zero temperature. This attenuates and spreads the harmonic peaks, obscuring the internal polaron transitions and giving closer agreement to the diagMC mobility at zero temperature. We have not used these methods here, but they will be investigated in future work to complement the multiple phonon model action with a more generalised trial action.

\section{Multiple Phonon Mode Dynamics}
\label{subsec:3-1-3}

To generalise the frequency-dependent mobility in Eq.~(\ref{eqn:freq_dep_mobility}), I follow the same procedure as FHIP but use our generalised polaron action $S$ in Eq.~(\ref{eqn:multiaction}) and trial action $S_0$ in Eq.~(\ref{eqn:multi_trial_action}). The result is a memory function akin to Eq.~(\ref{eqn:fhip_chi}) that is inclusive of multiple ($m$) phonon branches $j$ and multiple ($2n$) variational parameters $v_{p}$ and $w_{p}$,
\begin{equation} \label{eqn:multi_memory}
    \begin{gathered}
        \chi(\Omega) = \sum_{j=1}^m \frac{2\alpha_j\omega^2_j}{3\sqrt{\pi}} \int_0^{\infty} dt\ \left[1 - e^{i\Omega t}\right] \textrm{Im} S_j(t)
    \end{gathered}
\end{equation}
where
\begin{equation}
    S_j(\Omega) = \frac{\cos \left(\omega_j \left[t - i\hbar\beta/2\right]\right)}{\sinh (\hbar\omega_j\beta/2)} [G(t)]^{-3/2}
\end{equation}
where $G(t)$ is just $G(it)$ from Eq.~(\ref{eqn:multi_D}) rotated back to real-time to give a generalised version of $D(u)$ in Eq.~(\ref{eqn:D_FHIP}) from FHIP,

The new multiple-phonon frequency-dependent mobility $\mu(\Omega)$ is then obtained from the real and imaginary parts of the generalised $\chi$ using Eq.~(\ref{eqn:freq_dep_mobility}). The frequency-dependent mobility $\mu(\Omega)$ is obtained from the impedance using
\begin{equation}\label{eqn:freq_dep_mobility}
\begin{aligned}
    \mu(\Omega)^{-1} &= \frac{m_b}{e} \sum_j^m \omega_j \textrm{Re}\left\{z_j(\Omega)\right\} \\
    &= \frac{m_b}{e} \sum_j^m \omega_j \frac{\Omega^4 - 2\ \Omega^2\  \textrm{Re}\chi_j(\Omega) + |\chi_j(\Omega)|^2}{\Omega\ \textrm{Im}\chi_j(\Omega)}
\end{aligned}
\end{equation}
where $\chi_j(\Omega)$ is just the $j$th component of $\chi(\Omega)$. The limit that the frequency $\Omega \rightarrow 0$ gives the FHIP dc-mobility extended to multiple phonon modes,
\begin{equation}
    \mu^{-1}_{dc} = \frac{m_b}{e}\lim_{\Omega \rightarrow 0} \sum_{j=1}^m \omega_j \frac{\textrm{Im}\chi_j(\Omega)}{\Omega}
\end{equation}
since $\textrm{Re}\chi(\Omega = 0) = 0$.

\section{Multiple Fictitious Particle Dynamics}

To generalise the frequency-dependent mobility in Eq.~(\ref{eqn:mobility}), we follow the same procedure as FHIP but use our generalised polaron trial action $S_0$ in Eq.~(\ref{eqn:multi_trial_action}). The result is a memory function akin to FHIP's $\chi$ (Eq.~(35) in FHIP~\cite{feynman_mobility_1962}), but includes multiple ($2n$) variational parameters $v_{p}$ and $w_{p}$,
\begin{equation}\label{eqn:multichi}
    \begin{gathered}
        \chi(\Omega) = \frac{\alpha \omega_0^{2}}{3\sqrt{\pi}} \int_0^{\infty} dt\ \left[1 - e^{i\Omega t}\right] \textrm{Im} S(t)
    \end{gathered} .
\end{equation}
Here, 
\begin{equation}
    S(t) = D_{\omega_0}(t) [G(t)]^{-3/2} ,
\end{equation}
where $G(t)$ is $G(\tau = i t)$ from Eq.~(\ref{eqn:multi_D}) rotated back to real-time to give a generalised version of $D(u)$ in Eq.~(35c) in FHIP,
\begin{equation}
    \begin{gathered}
         G(t) = i t  \left(1 - \frac{i t}{\hbar\beta}\right) + \sum_{p=1}^n \frac{h_p}{v_p^3} \left(D_{v_p}(0) - D_{v_p}(\tau) - iv_p t\left(1 - \frac{it}{\hbar\beta} \right) \right).
    \end{gathered}
\end{equation}
The new frequency-dependent mobility $\mu(\Omega)$ is then obtained from the real and imaginary parts of the generalised $\chi(\Omega)$ using Eq.~(\ref{eqn:mobility}).

\subsection{The Effect on Dynamics}

\begin{figure}[!tbp]
    \centering
  \begin{subfigure}[b]{0.49\textwidth}
    \centering
    \includegraphics[width=\textwidth]{figures/cond_freq_bad.png}
  \end{subfigure}
  \hfill
  \begin{subfigure}[b]{0.49\textwidth}
    \centering
    \includegraphics[width=\textwidth]{figures/cond_alpha.png}
  \end{subfigure}
  \caption{Real component of the complex conductivity $\Re \sigma(\Omega)$ for the Fr\"ohlich model for increasing number $N$ of fictitious particles in the trial model. \textbf{Left:} The frequency-dependence of the real conductivity at $\alpha = 6$ and $\beta\omega_0=12.375$. Here I are in the regime for the maximum potential improvement on the trial model with the additional of fictitious particles. Despite minor improvements on the free energy approximation, the corresponding prediction for the real conductivity changes drastically due to the sensitivity of analytic continuation of the trial model. \textbf{Right:} Dependence of the ground-state ($T = 0$ K) real conductivity with the dimensionless electron-phonon coupling $\alpha$. The real conductivity converges quickly to its optimal solution as soon as $N=3$ with the maximal improvement occurring around $\alpha = 6$.}
  \label{fig:multidyn}
\end{figure}
Given that the multiple fictitious particle trial model rapidly converges to the optimal bound on the polaron-free energy, it is constructive to investigate how the dynamics of the trial system are altered. In Figs. (\ref{fig:multidyn}) the left figure shows the frequency-dependent conductivity at $\alpha = 6$ and thermodynamic temperature $\beta = 12.375 \omega_0$ for a number of fictitious particles coupled to the electron $N = 1, 2, 3$ and $4$. This is a regime where, from our previous observations, we expect to be close to the maximum potential improvement to the trial model by adding more fictitious particles. All four trial models agree at low frequencies below the phonon frequency. However, upon reaching the phonon frequency and beyond, we see that each trial model produces a conductivity with different oscillation periods and amplitudes. The frequency of this oscillation is smallest for the $N=1$ trial model and increases with each additional particle. Meanwhile, the amplitude of the oscillation decreases with more particles. The right figure shows the coupling dependence of the conductivity at zero temperature for $N=1, 2,3$ and $4$. This figure shows that the maximum improvement in the zero-temperature conductivity is around $\alpha = 6$ with little discernible difference between $N=3$ and $4$. Despite the apparent convergence of the conductivity for $N=3$ and $N=4$, the frequency dependence shows a significant difference, suggesting that far more fictitious particles may be required to reach a truly converged optimal solution in the frequency response of the trial system.

\section{The Holstein Model}

\subsection{Polaron Mobility}

\begin{figure}[!tbp]
    \includegraphics[width=.49\textwidth]{figures/holstein-2d-mobility-temp-00625to32-COLOUR.pdf}
    \includegraphics[width=.49\textwidth]{chapters/responsefunctions/figures/frohlich-2d-mobility-temp-00625to32-COLOUR.pdf}
    \includegraphics[width=.49\textwidth]{chapters/responsefunctions/figures/holstein-3d-mobility-temp-00625to32-COLOUR.pdf}
    \includegraphics[width=.49\textwidth]{chapters/responsefunctions/figures/frohlich-3d-mobility-temp-00625to32-COLOUR.pdf}
    \includegraphics[width=.49\textwidth]{figures/holstein-1d-mobility-temp-00625to32-COLOUR.pdf}
    \caption{Temperature dependence ($T$, in units of phonon frequency $\omega_0$) of the polaron DC mobility $\mu$ for the Fr\"ohlich model in 2D (solid blue) and 3D (dashed orange), and for the Holstein model in 1D (dot-dashed green), 2D (dot-dot-dashed pink) and 3D (solid gold), for values of the Fr\"ohlich electron-phonon coupling $\alpha = 2.5, 4, 6, 8, 10, 12$ and $1/3$ of these values for the Holstein electron-phonon coupling.}
    \label{fig:mobility_temp}
\end{figure}
In Figs. (\ref{fig:mobility_temp}) we have the temperature dependence of the polaron mobility for the Holstein polaron with varying electron-phonon coupling. At weaker coupling, mobility shows the typical exponentially decreasing band-like transport at temperatures below the phonon energy level. Above the phonon energy, the temperature dependence transitions to a power-law relationship $T^{-x}$ where $x$ is some number typically used to determine the dominant scattering mechanism within a material. For example, this index is typically $x = 3/2$ for acoustic phonons. 

Again, as we saw previously, 2D Holstein and 3D Fr\"ohlich appear to be most alike. As the electron-phonon coupling increases, we begin to see the onset of the ski-slope feature where the mobility takes on a local minimum at the phonon energy $T = \omega_0$ before increasing to a local maximum at the polaron quasiparticle frequency $v$ and then transitioning back into a power-law relationship at higher temperatures. Each of the different dimensions of both polaron models seems to have a distinct dependence on the strength of the electron-phonon coupling when it comes to mobility. For the Fr\"ohlich polaron, the ski-slope appears sooner for the 2D model than the 3D model. For the Holstein polaron, the opposite trend seems accurate, with the higher dimension model exhibiting the ski slope. The 2D Fr\"ohlich polaron also appears to manifest a low-temperature maximum at larger electron-phonon couplings, which transitions in a linear decrease to some finite value in mobility as the temperature goes to zero.

At high temperatures, the Fr\"ohlich polaron mobility follows the temperature power-law $\mu^{(H)} \sim T^{-1/2}$, which is consistent with mobility derived from the electron scattering with optical phonons. However, the Holstein polaron mobility becomes constant at high temperatures, independent of temperature. This may be attributed to the phonon-induced electron hopping between lattice sites along which the electron motion is coherent in the direction of that particular energy band. Likewise, at low temperatures, the mobility increases abruptly below the Debye temperature due to the increasing contribution of the electron transfer without phonon participation.

\section{Frequency-Dependent Optical Conductivity}
\label{sec:5-2}

\subsection{Holstein Model}
\label{subsec:5-2-1}

This section examines how the Holstein model varies with an external, perturbing, and varying electric field frequency. Specifically, the polaron memory function, which we compare to the Fr\"ohlich polaron results of FHIP~\cite{feynman_mobility_1962}, and the optical conductivity, which we compare to the Fr\"ohlich polaron results of DSG~\cite{devreese_optical_1972}. For the memory function we look at frequencies $\Omega / \omega_0$ from $0$ to $28$ and $\alpha^{(F)} = 3, 5, 7$ or $\alpha^{(H)} = 1, 1.67, 2.33$. For the optical conductivity we look at frequencies $\Omega / \omega_0$ from $0$ to $11$ for $\alpha^{(F)} = 1, 3, 5, 6$ or $\alpha^{(H)} = 0.33, 1, 1.67, 2$ and frequencies $\Omega / \omega_0$ from $0$ to $22$ for $\alpha^{(F)} = 7$ or $\alpha^{(H)} = 2.33$. 

\subsubsection{Polaron Memory Function}

\begin{figure}
\centering
  \begin{subfigure}[b]{0.49\textwidth}
    \includegraphics[width=\textwidth]{figures/holstein-1d-imag-memory-freq-COLOUR.pdf}
  \end{subfigure}
  \begin{subfigure}[b]{0.49\textwidth}
    \includegraphics[width=\textwidth]{figures/holstein-2d-imag-memory-freq-COLOUR.pdf}
  \end{subfigure}
  \begin{subfigure}[b]{0.49\textwidth}
    \includegraphics[width=\textwidth]{figures/holstein-3d-imag-memory-freq-COLOUR.pdf}
  \end{subfigure}
  \begin{subfigure}[b]{0.49\textwidth}
    \includegraphics[width=\textwidth]{figures/frohlich-3d-imag-memory-freq-COLOUR.pdf}
  \end{subfigure}
  \caption{Frequency dependence ($\Omega$, in units of the phonon frequency $\omega_0$) of the polaron memory function $\chi(\Omega)$ for the Fr\"ohlich model in 3D (solid blue), and the Holstein model in 1D (dashed orange), 2D (dot-dashed green) and 3D (solid gold), for values of the Fr\"ohlich electron-phonon coupling $\alpha = 1, 3, 5,6, 7$ and $1/3$ of these values for the Holstein electron-phonon coupling. Here, I only consider the imaginary component to reduce graph clutter, but the real component was also evaluated.} 
  \label{fig:im_mem_freq}
\end{figure}
In Figs. (\ref{fig:im_mem_freq}) is the frequency-dependent imaginary component of the memory function for the Holstein and Fr\"ohlich polarons for a range of electron-phonon coupling strengths. I chose these specific alpha values $\alpha^{(F)} = 3, 5, 7$ for direct comparison to figures (1-3) in FHIP~\cite{feynman_mobility_1962}. Note that here we use Devreese and Peeter's definition of the memory function $\Sigma(\Omega)$~\cite{peeters_theory_1984}, which is related to the FHIP memory function $\chi(\Omega)$ by the expression $\Sigma(\Omega) = \chi(\Omega) / \Omega$. I have excluded the 2D Fr\"ohlich result as even at `weak' coupling, it behaves like the 3D Fr\"ohlich result at strong coupling and is difficult to compare.

The first most noticeable observation is that the 1D Holstein memory function seems to most closely relate to the 3D Fr\"ohlich memory function. This also is not surprising because the $q$-space integral in the memory function $\Sigma(\Omega)$ is the same for the 3D Fr\"ohlich and 1D Holstein polarons. Additionally, the imaginary component of the memory function is zero for frequencies below the phonon frequency, and the first peak corresponds to one phonon excitation.

As the electron-phonon coupling strength increases, more oscillations manifest at multiples of the polaron quasiparticle frequency $\Omega_{\text{peaks}} = 1 + n v, n \in \mathbf{N}$ which correspond to two- three- four etc phonon excitations. These peaks are significantly more substantial in the Holstein polaron. 

The 2D and 3D Holstein memory functions take very different forms from what is seen for the 1D Holstein and 3D Fr\"ohlich polarons. At lower coupling, there seems to be more of a background lattice response that obscures the underlying phonon excitations until the small polaron state is formed at $\alpha^{(H)} > 2$.

\subsubsection{Polaron Optical Conductivity}

\begin{figure}[!tbp]
    \centering
    \includegraphics[width=.49\textwidth]{figures/holstein-1d-real-conductivity-freq-COLOUR.pdf}
    \includegraphics[width=.49\textwidth]{figures/holstein-2d-real-conductivity-freq-COLOUR.pdf}
    \includegraphics[width=.49\textwidth]{figures/holstein-3d-real-conductivity-freq-COLOUR.pdf}
    \includegraphics[width=.49\textwidth]{figures/frohlich-3d-real-conductivity-freq-COLOUR.pdf}
    \caption{Frequency dependence ($\Omega$, in units of the phonon frequency $\omega_0$) of the polaron complex conductivity $\sigma(\Omega)$ for the Fr\"ohlich model in 2D (solid blue) and 3D (dashed orange), and for the Holstein model in 1D (dotted green), 2D (dot-dashed pink) and 3D (d pink), for values of the Fr\"ohlich electron-phonon coupling $\alpha = 3, 5, 7$ and $1/3$ of these values for the Holstein electron-phonon coupling. Here, I only consider the imaginary component to reduce graph clutter, but the real component was also evaluated.}
    \label{fig:re_con_freq}
\end{figure}
In Figs. (\ref{fig:re_con_freq}) is the frequency-dependent real component of the complex conductivity (otherwise known as the optical conductivity) for the Holstein and Fr\"ohlich polarons for a range of electron-phonon coupling strengths. I chose these specific alpha values $\alpha^{(F)} = 1, 3, 5, 6, 7$ for direct comparison to Figs.~(1-5) in DSG~\cite{devreese_optical_1972}.

Starting at weak coupling $\alpha^{(F)} = 1, \alpha^{(H)} = 0.33$, both models in all the presented spatial dimensions show the same form of a one-phonon excitation peak that decays away at higher frequencies. At intermediate coupling, we begin to see more structure. Firstly, the 2D Fr\"ohlich polaron shows many peaks and side bands only made apparent in the 3D Fr\"ohlich polaron at strong coupling. At $\alpha^{(F)} = 3$, we already see a very intense relaxed excited state (RES) transition occurs for Fr\"ohlich 2D followed by a prominent Frank-Condon (FC) excitation peak. Conversely, the other Fr\"ohlich 3D and Holstein 1D, 2D and 3D only exhibit the initial one-phonon peak with no apparent RES transitions or FC peaks. At $\alpha^{(F)} = 5, \alpha^{(H)} = 1.67$, the features of the 2D Fr\"ohlich polaron are pushed to very high frequencies far beyond the other polarons. However, the 3D Fr\"ohlich polaron only just begins to develop a RES transition peak around $\Omega = v$ with the shoulder on the low-frequency side representing the original one-phonon peak and the additional peak on the high-frequency side representing a FC band. The Holstein polaron only possesses the one-phonon peak for 1D, 2D and 3D. This pattern continues until we exceed $\alpha^{(H)} = 2$. Unlike the Fr\"ohlich polaron, which developed RES and FC peaks before its polaron transition, the Holstein polaron does not establish these features until the coupling increases beyond $\alpha^{(H)} = 2$ where a small polaron state is formed, at which point the onset of strong RES and FC states is far more rapid than for the Fr\"ohlich polaron. 

\section{General Polaron Mobility}
\label{sec:chap-fifth-first}

The polaron DC mobility may be obtained in the same way as was done for the Fr\"ohlich model from real components of the frequency- and temperature-dependent impedance function,
\begin{equation}\label{eqn:mobility}
    \mu_{dc} = \lim_{\Omega \to 0} \Re{\frac{1}{z(\Omega)}},
\end{equation}
where the impedance function is expressed in terms of the memory function $\Sigma(\Omega)$,
\begin{equation}
    z(\Omega) = i \left( \Omega - \Sigma(\Omega) \right).
\end{equation}
More specifically, we can express the inverse DC mobility just in terms of the memory function,
\begin{equation}
    \mu_{dc}^{-1} = \lim_{\Omega \to 0} \Im{\Sigma(\Omega)}.
\end{equation}
Starting from expressing the dynamic memory function for a general polaron and specialising in the Holstein case, the general memory function can be written as,
\begin{equation}
    \Sigma(\Omega) = \frac{4}{n m_b \hbar\Omega} \int_0^{\infty} dt\ \left(1 - e^{i \Omega t}\right) \Im \left\{ \sum_{\vb{q}} \abs{V_{\vb{q}}}^2 q^2 D_{\omega_{\vb{q}}}(t) \langle e^{i \vb{q} \cdot \left[ \vb{r}(t) - \vb{r}(0) \right]} \rangle \right\},
\end{equation}
where we assume the system to be rotationally invariant. Here $D_{\omega}(t)$ is the \emph{real}-time thermal and \emph{dynamical} phonon Green function,
\begin{equation}
    D_\omega(t) = \coth(\frac{\hbar \beta \omega}{2}) \cos(\omega t) - i \sin(\omega t),
\end{equation}
and can be obtained from substituting $\tau \to i t$ into Eq.~(\ref{eqn:phonongf}).

The key term to evaluate is the density-density correlation function or dynamical structure factor,
\begin{equation}
    S_{\vb{q}}(t) = \langle \rho_{\vb{q}}(t) \rho^*_{\vb{q}}(0) \rangle = \langle e^{i \vb{q} \cdot \left[ \vb{r}(t) - \vb{r}(0) \right]} \rangle.
\end{equation}
As done earlier, the expectation value can be expressed as a path integral. However, even the full Fr\"ohlich model cannot be evaluated exactly. We approximate this by evaluating this expectation value with respect to the Feynman Polaron model,
\begin{equation}
    \langle \rho_{\vb{q}}(t) \rho^*_{\vb{q}}(0) \rangle_0 = e^{-q^2 r_p^2 G(t)},
\end{equation}
where $G(t)$ is the polaron Green function evaluated in real-time (i.e. substitute $\tau \to it$ into Eq.~(\ref{eqn:polarongreensfunc}),
\begin{equation}
    G(t) = i t \left(1 - \frac{i t}{\hbar \beta} \right) + \frac{v^2 - w^2}{v^3} \left[ D_v(0) - D_v(t)  - i v t \left(1 - \frac{i t}{\hbar \beta} \right) \right].
\end{equation}
In $n$-dimensions, we need to evaluate the reciprocal-space integral,
\begin{equation}
    \begin{aligned}
        I(n) &= V_0 \int \frac{d^n q}{(2\pi)^n} \abs{M_{\vb{q}}}^2 q^2 D_{\omega_{\vb{q}}}(t) e^{-q^2 r_p^2 G(t)}, \\
        &= \frac{V_0 \abs{S^{n-1}}}{(2\pi)^n} \int_0^R dq\ q^{n+1} \abs{V_{q}}^2 D_{\omega_{q}}(t) e^{-q^2 r_p^2 G(t)} ,
    \end{aligned}
\end{equation}
where we have used that the system is rotation-invariant. For a general polaron model, we then have the memory function,
\begin{equation}
    \Sigma(\Omega) = \frac{4}{n m_b\hbar\Omega} \frac{V_0 \abs{S^{n-1}}}{(2\pi)^n} \int_0^\infty dt\ \left(1 - e^{i \Omega t}\right) \int_0^\Lambda dq\ \abs{M_{q}}^2 q^{n+1} \Im{D_{\omega_q}(t) e^{-q^2 r_p^2 G(t)}}.
\end{equation}
In the zero frequency limit, we have,
\begin{equation}
    \lim_{\Omega \to 0} \frac{\left(1 - e^{i \Omega t}\right)}{\Omega} \to -i t,
\end{equation}
so for the general polaron DC mobility, we get,
\begin{equation}
    \mu_{dc}^{-1} = -\frac{4 e^2}{n m_b \hbar} \frac{V_0 \abs{S^{n-1}}}{(2\pi)^n} \int_0^\infty dt\ t \int_0^\Lambda dq\ q^{n+1} \abs{M_{q}}^2 \Im{ D_{\omega_{q}}(t) e^{-q^2 r_p^2 G(t)}} .
\end{equation}

\subsection{Specialising to the Fr\"ohlich Model}

Now equipped with the general polaron variational equations for the free energy and the corresponding memory function, we can specialise in the Fr\"ohlich model by substituting
\begin{subequations}
    \begin{equation}
        \abs{M_{\vb{q}}}^2 = g^2_F(n) / V q^{n-1},
    \end{equation}
    \begin{equation}
        \omega_{q} = \omega_0,
    \end{equation}
    \begin{equation}
         \Lambda \to \infty.
    \end{equation}
\end{subequations}
The $q$-space integral is evaluated,
\begin{equation}
    \begin{aligned}
    I^{(F)}(n) &= \frac{g^2_F(n) \abs{S^{n-1}}}{(2\pi)^n} D_{\omega_0}(t) \int_0^\infty dq\ q^2 e^{-q^2 r_p^2 G(t)}, \\
    &= \frac{g_F^2(n) \abs{S^{n-1}} \sqrt{\pi}}{(2\pi)^n 4 r_p^3} \frac{D_{\omega_0}(t)}{G(t)^{\frac{3}{2}}}, \\
    &= \alpha^{(F)} \frac{\sqrt{\pi}}{2 r_p^3} \frac{(2 \sqrt{\pi})^{-n}}{\Gamma\left(\frac{n}{2}\right)} \frac{D_{\omega_0}(t)}{G(t)^{\frac{3}{2}}} .
    \end{aligned}
\end{equation}
The memory function for the Fr\"ohlich model is then,
\begin{equation}
    \Sigma^{(F)}(\Omega) =  \frac{1}{m_b \hbar \Omega r_p^3} \frac{\pi \sqrt{2 \pi} \alpha^{(F)}}{\Gamma\left(\frac{n}{2} + 1\right) \left(2 \sqrt{\pi}\right)^n} \int_0^\infty dt\ \left(1 - e^{i \Omega t}\right) \frac{D_{\omega_0}(t)}{\left[ G(t)\right]^{3/2}}.
\end{equation}
and the inverse Fr\"ohlich DC mobility is,
\begin{equation}
    \mu_{dc}^{-1} = -\frac{e^2}{m_b \hbar r_p^3} \frac{\pi \sqrt{2 \pi} \alpha^{(F)}}{\Gamma\left(\frac{n}{2} + 1\right) \left(2 \sqrt{\pi}\right)^n} \int_0^\infty dt\ \frac{t D_{\omega_0}(t)}{\left[ G(t)\right]^{3/2}}.
\end{equation}

\subsection{Specialising to the Holstein Model}

We can specialise in the Holstein model by substituting,
\begin{subequations}
    \begin{equation}
        \abs{M_{\vb{q}}}^2 = g_H^2(n) / N ,
    \end{equation}
    \begin{equation}
        \omega_{\vb{q}} = \omega_0 ,
    \end{equation}
    \begin{equation}
        \Lambda = 2\sqrt{\pi} \left(V \Gamma\left(\frac{n}{2} + 1\right)\right)^{1/n} \equiv \Lambda_n .
    \end{equation}
\end{subequations}
The $q$-space integral is evaluated,
\begin{equation}
    \begin{aligned}
    I^{(H)}(n) &= \frac{g^2_H(n) \abs{S^{n-1}}}{(2\pi)^n} D_{\omega_0}(t) \int_0^{\Lambda_n} dq\ q^{n+1} e^{-q^2 r_p^2 G(t)} , \\
    &= \frac{g^2_H(n) \abs{S^{n-1}}}{2 (2\pi)^n r_p^{n+2}}
    \frac{D_{\omega_0}(t)}{G(t)^{\frac{n}{2}+1}} \left[ \Gamma\left(\frac{n}{2} +1\right) - \Gamma\left(\frac{n}{2}+1, r_p^2 \Lambda_n^2 G(t) \right) \right] , \\
    &= \frac{1}{2 r_p^2} \frac{n g_H^2(n)}{(2 r_p \sqrt{\pi})^n} \frac{D_{\omega_0}(t)}{G(t)^{\frac{n}{2}+1}} \left[ 1 - \frac{\Gamma\left(\frac{n}{2} + 1, r_p^2 \Lambda_n^2 G(t) \right)}{\Gamma\left(\frac{n}{2} + 1\right)} \right] .
    \end{aligned}
\end{equation}
The memory function for the Holstein model is then:
\begin{equation}
    \Sigma^{(H)}(\Omega) = \frac{1}{m_b\hbar\Omega \gamma r_p^{n+2}} \frac{4 n \alpha^{(H)}}{(2 \sqrt{\pi})^n} \int_0^\infty dt\ \left(1 - e^{i \Omega t}\right) \frac{D_{\omega_0}(t)}{G(t)^{\frac{n}{2}+1}} \left[ 1 - \frac{\Gamma\left(\frac{n}{2} + 1, r_p^2 \Lambda_n^2 G(t) \right)}{\Gamma\left(\frac{n}{2} + 1\right)} \right],
\end{equation}
where $\gamma = \hbar\omega_0 / J$ is the adiabaticity. The inverse Holstein DC mobility is then,
\begin{equation}
    \mu_{dc}^{-1} = \frac{e^2}{m_b\hbar \gamma r_p^{n+2}} \frac{4 n \alpha^{(H)}}{(2 \sqrt{\pi})^n} \int_0^\infty dt\ t \frac{D_{\omega_0}(t)}{G(t)^{\frac{n}{2}+1}} \left[ 1 - \frac{\Gamma\left(\frac{n}{2} + 1, r_p^2 \Lambda_n^2 G(t) \right)}{\Gamma\left(\frac{n}{2} + 1\right)} \right].
\end{equation}

\section{The Optimal Self-Consistent Response Function}
\label{sec:chap-fifth-third}

\subsection{Statics = Dynamics}

The function $\Sigma(\Omega)$, which gives the lowest energy in the variational principal at zero temperature, satisfies the integral equation,
\begin{equation}
    \Sigma(\Omega) = \frac{2}{n} \sum_{\vb{q}} \abs{V_{\vb{q}}}^2 q^2 \int^\infty_0 d\tau \left( 1 - \cos(\Omega \tau) \right)  e^{-\omega_{\vb{q}} \tau} e^{-q^2 r_p^2 [G(0) - G(\tau)]}, 
\end{equation}
where,
\begin{equation}
    G(\tau) = \int_{-\infty}^\infty \frac{d\omega}{2\pi} \frac{ e^{i \omega \tau} }{m \omega^2 - \Sigma(\omega)} .
\end{equation}
The polaron Green's function $G(\tau)$ that minimises the polaron free energy at arbitrary temperature without applied electric and magnetic fields also produces the optimal impedance function $z(\Omega)$. 

Generalising the variational equations to a general polaron system, we obtain,
\begin{equation}
    F \leq \frac{n}{\beta} \sum_{l=1}^\infty \ln \left( \frac{Z(\omega_l)}{m \omega_l^2} \right) - \frac{n}{\beta} \sum_{l=1}^\infty \frac{1 - m \omega^2_l}{Z(\omega_l)} - \int_0^{\frac{\hbar\beta}{2}} d\tau \sum_{\vb{q}, j} \abs{V_{\vb{q}, j}}^2 D_{\vb{q}, j}(\tau) e^{-q^2 r_p^2 \left[G(0) - G(\tau)\right]},
\end{equation}
where,
\begin{equation}
    \Sigma(\tau) = \frac{2}{\beta} \sum_{l = 1}^\infty \frac{1 - \cos\left(\omega_l \tau \right)}{Z(\omega_l)},
\end{equation}
and,
\begin{equation}
    Z(\omega_l) = m \omega_l^2 + 4 \int_{-\infty}^\infty d\Omega \frac{P}{\Omega} \frac{G(\Omega) \omega^2_l}{\Omega^2 + \omega_l^2},
\end{equation}
and $\omega_l \equiv 2\pi l / \beta$, $l \in \mathbf{N}$.