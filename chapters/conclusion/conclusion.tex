\chapter{Conclusion \& Outlook}
\label{chap:seventh}

\thesisepisrcyear{What is the meaning of life?}{John Doe}{Thoughts}{1971}

\chapterintrobox{This is the introduction paragraph.}

% During my PhD, I have focused on improving the Feynman path integral variational method and apply it to more general model systems that better represent realistic materials. Feynman's original application of his variational method was to the Fr\"ohlich large polaron model, where his trial action was chosen to represent the electron harmonically-coupled via a spring to some fictitious particle which represents the many phonons. Here the mass and spring constant (or equivalently the frequency) of the trial model act as two variational parameters which, when combined with the Feynman-Jensen inequality, can be varied to find the optimal upper-bound to the true free energy. The main focus of this LSR was on applying this method to the Holstein lattice-polaron Hamiltonian - something that has not been done before. The motivation for doing so is that this Hamiltonian is thought to be a more faithful representation of the physics within organic crystal materials often used in modern electronic devices. Here I have derived the path integral corresponding to the Holstein model, discussed some of the challenges for applying Feynman's variational method (due to the non-quadratic nature of the electron kinetic action and constrained electronic paths) and presented some approximations to produce a similar parabolic-band Hamiltonian that we can use the variational method on, but which still maintains a lattice description for the phonons in the model. I then applied the newly derived variational Holstein model to the organic semiconductor Rubrene, for which I have derived its polaronic properties (polaron binding energy, size, effective mass etc.) and dynamical properties (charge-carrier mobility, complex conductivity etc.), and compared to those derived from the original variational Fr\"ohlich large-polaron model. I found that the new variational Holstein model predicts more realistic values for the Rubrene polaron than the variational Fr\"ohlich model, especially for the room-temperature mobility as shown in Tables 6.2 and 6.3.
% \newline

% In addition to developing the variational Holstein model, I also discussed how to more generally improve the Feynman variational method and derived an extended trial model of an electron coupled to a finite number of fictitious particles. Generally, there are three main ways to improve this variational method: extend its applicability to more descriptive Hamiltonian models that better capture real material properties; improve the trial model; or extend the Feynman-Jensen inequality to include higher-order cumulants. For the first possibility, during my PhD I have investigated including multiple phonon modes in Fr\"ohlich's polaron model \cite{Martin2022}, including anisotropy in the electron-band \cite{Bogdan2021} and including general phonon-momentum dependence. Another researcher, Matthew Houtput, has even generalised this method to a model with anharmonic phonons \cite{Houtput2021}. I have also investigated applying this method to the Holstein lattice-polaron model with some success in \cite{MartinArxiv2022} and this LSR. For the second possibility, I have extended the spring-mass trial model to include multiple fictitious masses in this LSR. In the limit of an infinite number of fictitious masses, we instead couple the electron to a fictitious harmonic bath where the choice of variational parameters is replaced with choice of the spectral function for the bath. The optimal spectral function can then be obtained self-consistently using functional methods \cite{Adamowski1980, Adamowski1984, Dries2016, Ichmoukhamedov2022}. For the final possibility, I have not investigated including higher-order cumulant corrections to the Feynman-Jensen inequality, although it has been investigated elsewhere \cite{Marshall1970, Rosenfelder1992}. There it was found that for the Fr\"ohlich model, the second-order cumulant corrects the $\alpha^2$ coefficient at weak-coupling (c.f. Eqn. (\ref{eqn:weakcoupling})), but offers minimal improvement at intermediate coupling. For other model Hamiltonians, including the second-order cumulant proved important for accurate results \cite{Ichmoukhamedov2022}. 
% \newline

% Interestingly, there is a connection \cite{Thornber1970,Thornbur1971} between this optimal spectral function and the memory function used in FHIP \cite{Feynman1962} (a Wick-rotation that can be performed since the function is analytic across a semi-infinite strip of the complex-plane) used to derive dynamical properties of the system, such as the complex conductivity and charge-carrier mobility. Furthermore, comparison of the free energy inequality (Eqn. (\ref{eqn:scfreeenergy})) with the expression for the Luttinger-Ward function \cite{Luttinger1960} suggests that the memory function and spectral function are comparable to the non-local one-particle self-energy function in real- or imaginary-time respectively. Likewise, the Feynman variational method may then be interpreted as an approximation to the Luttinger-Ward-like functional where one either variationally (for a convex path integrand) or self-consistently seeks the optimal non-local trial self-energy function, or equivalently the optimal trial one-particle Green function. Although not directly comparable, this is conceptually similar to approximating the exchange-correlation functional in Density Functional Theory (DFT) and self-consistently solving for the optimal electron density that minimises the ground-state energy. I hypothesise that the form of the trial Green function obtained from the Feynman variational method may have a connection to that obtained from cumulant Green function methods \cite{Whitefield1963,Lundqvist1969,Langreth1970}. If this connection can be consolidated, it may open up an opportunity for generalising the Feynman variational method to many-body Hamiltonians (e.g. many electrons rather than one) or even grand canonical systems with a variable number of particles, by using the toolkit of many-body Green functions \cite{Giustino2017} via the analogies I have identified. I think that a next logical step would be to derive the variational method for arbitrary representations of the path integral, such as the phase-space representation for canonical ensembles, and coherent state path integrals for grand-canonical ensembles \cite{Altland_Simons_2010}. This way, we may be able to go beyond the parabolic-band approximation for the electron (or other particles) to general electron band-structures. This may remove the discrepancy found between the predicted polaron binding energy at higher coupling using the derived variational Holstein method compared to those calculated using Diagrammatic Monte Carlo (as shown in Fig. 6.1).
% \newline 

% The `ultimate' vision for this project would be to have a highly accurate, potentially ab-initio, variational method for predicting both thermodynamic and dynamical properties of real complex materials and to apply it across whole classes of materials to discover trends that would aid material design and discovery. The attractiveness of this method over other methods, such as Monte-Carlo methods like Path Integral or Diagrammatic Monte Carlo, or purely self-consistent methods such as GW-theory or DMFT, is that this variational path integral method provides an efficient framework for generalising to many different quantum systems by making use of powerful toolkits from Quantum Field Theory, Functional variation and even potentially Many-Body Green Functions. Likewise, it has controlled errors, unlike Monte-Carlo, and has a well defined variational bound due to the convex nature of the path integrand, allowing for computationally tractable optimisation. 
% \newline

% However, I would be amiss to neglect a glaring limitation and potential problem with the variational method. In my analysis of the multiple fictitious particle trial action, I showed in Fig. 6.13 that the higher frequency dynamics of the system can be extremely sensitive to the derived optimal ground-state of the thermodynamic system. This was also investigated by Dries Sels \cite{Dries2016}. The cause of this discrepancy is due to the limited \emph{physics} of the trial system. Even the most general non-local quadratic action (Eqn. (\ref{eqn:generaltrial})) is still quadratic in its variables. In fact, a quadratic action indicates a lack of true particle-particle interactions in the theory as quadratic actions are typically associated with free non-interacting theories; the self-energy term here merely acts as a temporally non-local external potential. The problem can be made clearer by analogy to the Luttinger-Ward functional. As I discussed, the optimal trial action in the variational method amounts to an approximate truncation of the diagrams that compose a Luttinger-Ward-like functional (the sum of all closed, bold two-particle irreducible diagrams). I am not sure which diagrams precisely that the optimal Feynman variational method keeps, yet I aim to understand this for my thesis. Conceptually, I think it would be similar to the diagrams kept in cumulant Green function methods, so the variational method includes an infinite number of diagrams likely to higher and higher order in phonons interacting with the electron, but not all possible diagrams. Nonetheless, we still need to keep the trial action quadratic as the resultant path integral needs to be soluble for the variational method to have any merit. I mentioned previously that the Feynman variational method can be improved by including the second-order correction in the Feynman-Jensen inequality. I think this would then include another infinite set of diagrams which were previously absent, improving the overall approximation. We could continue in this way to improve the approximated Luttinger-Ward-like functional, adding more infinite sets of electron-phonon diagrams, where the optimal electron Green function is to be solved self-consistently. Obviously, the computational complexity of this method would be very limiting, which bring me to my next idea; machine learning.
% \newline

% Ultimately, what the Feynman-Jensen inequality does is map a complicated, intractable probability distribution to a simpler, optimal Gaussian distribution via a change of variable - albeit in an infinite-dimensional function space. This is a form of variational inference \cite{Neal1992}. The free energy inequality then has a \emph{functional} form of the Kullback-Leibler divergence often used in machine learning, which measures the statistical difference between the two continuous functional distributions. In the Feynman variational method, the Kullback-Leibler divergence is truncated to include only the first cumulant in the difference between the two distributions. A natural extension to higher order cumulants points to the use of \emph{normalising flows} \cite{Kobyzev2021}. So, instead of trying to solve the optimal Luttinger-Ward-like functional self-consistently, we could replace it with a series of deep neural-networks which are trained to minimise the Kullback–Leibler divergence between the trial model's likelihood and the true target distribution. The training data can be obtained by sampling directly from the target distribution as done for Path Integral Monte Carlo. A nice connection is that in the limit to a \emph{continuous} normalising flow, optimal-control methods may be used which can themselves be formulated in terms of Wiener stochastic path integrals and the Feynman-Kac formula. I do not know if I would have time to develop and code this machine learning method during my PhD or even if it have any practical merit, but I would at least like to include some theory behind it towards the end of my thesis.