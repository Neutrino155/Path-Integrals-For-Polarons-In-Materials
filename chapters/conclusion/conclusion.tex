\chapter{Conclusion \& Outlook}
\label{chap:seventh}

\thesisepisrcyear{What is the meaning of life?}{John Doe}{Thoughts}{1971}

\chapterintrobox{This is the introduction paragraph.}

% During my PhD, I have focused on improving the Feynman path integral variational method and apply it to more general model systems that better represent realistic materials. Feynman's original application of his variational method was to the Fr\"ohlich large polaron model, where his trial action was chosen to represent the electron harmonically-coupled via a spring to some fictitious particle which represents the many phonons. Here the mass and spring constant (or equivalently the frequency) of the trial model act as two variational parameters which, when combined with the Feynman-Jensen inequality, can be varied to find the optimal upper-bound to the true free energy. The main focus of this LSR was on applying this method to the Holstein lattice-polaron Hamiltonian - something that has not been done before. The motivation for doing so is that this Hamiltonian is thought to be a more faithful representation of the physics within organic crystal materials often used in modern electronic devices. Here I have derived the path integral corresponding to the Holstein model, discussed some of the challenges for applying Feynman's variational method (due to the non-quadratic nature of the electron kinetic action and constrained electronic paths) and presented some approximations to produce a similar parabolic-band Hamiltonian that we can use the variational method on, but which still maintains a lattice description for the phonons in the model. I then applied the newly derived variational Holstein model to the organic semiconductor Rubrene, for which I have derived its polaronic properties (polaron binding energy, size, effective mass etc.) and dynamical properties (charge-carrier mobility, complex conductivity etc.), and compared to those derived from the original variational Fr\"ohlich large-polaron model. I found that the new variational Holstein model predicts more realistic values for the Rubrene polaron than the variational Fr\"ohlich model, especially for the room-temperature mobility as shown in Tables 6.2 and 6.3.
% \newline

% In addition to developing the variational Holstein model, I also discussed how to more generally improve the Feynman variational method and derived an extended trial model of an electron coupled to a finite number of fictitious particles. Generally, there are three main ways to improve this variational method: extend its applicability to more descriptive Hamiltonian models that better capture real material properties; improve the trial model; or extend the Feynman-Jensen inequality to include higher-order cumulants. For the first possibility, during my PhD I have investigated including multiple phonon modes in Fr\"ohlich's polaron model \cite{Martin2022}, including anisotropy in the electron-band \cite{Bogdan2021} and including general phonon-momentum dependence. Another researcher, Matthew Houtput, has even generalised this method to a model with anharmonic phonons \cite{Houtput2021}. I have also investigated applying this method to the Holstein lattice-polaron model with some success in \cite{MartinArxiv2022} and this LSR. For the second possibility, I have extended the spring-mass trial model to include multiple fictitious masses in this LSR. In the limit of an infinite number of fictitious masses, we instead couple the electron to a fictitious harmonic bath where the choice of variational parameters is replaced with choice of the spectral function for the bath. The optimal spectral function can then be obtained self-consistently using functional methods \cite{Adamowski1980, Adamowski1984, Dries2016, Ichmoukhamedov2022}. For the final possibility, I have not investigated including higher-order cumulant corrections to the Feynman-Jensen inequality, although it has been investigated elsewhere \cite{Marshall1970, Rosenfelder1992}. There it was found that for the Fr\"ohlich model, the second-order cumulant corrects the $\alpha^2$ coefficient at weak-coupling (c.f. Eqn. (\ref{eqn:weakcoupling})), but offers minimal improvement at intermediate coupling. For other model Hamiltonians, including the second-order cumulant proved important for accurate results \cite{Ichmoukhamedov2022}. 
% \newline

% Interestingly, there is a connection \cite{Thornber1970,Thornbur1971} between this optimal spectral function and the memory function used in FHIP \cite{Feynman1962} (a Wick-rotation that can be performed since the function is analytic across a semi-infinite strip of the complex-plane) used to derive dynamical properties of the system, such as the complex conductivity and charge-carrier mobility. Furthermore, comparison of the free energy inequality (Eqn. (\ref{eqn:scfreeenergy})) with the expression for the Luttinger-Ward function \cite{Luttinger1960} suggests that the memory function and spectral function are comparable to the non-local one-particle self-energy function in real- or imaginary-time respectively. Likewise, the Feynman variational method may then be interpreted as an approximation to the Luttinger-Ward-like functional where one either variationally (for a convex path integrand) or self-consistently seeks the optimal non-local trial self-energy function, or equivalently the optimal trial one-particle Green function. Although not directly comparable, this is conceptually similar to approximating the exchange-correlation functional in Density Functional Theory (DFT) and self-consistently solving for the optimal electron density that minimises the ground-state energy. I hypothesise that the form of the trial Green function obtained from the Feynman variational method may have a connection to that obtained from cumulant Green function methods \cite{Whitefield1963,Lundqvist1969,Langreth1970}. If this connection can be consolidated, it may open up an opportunity for generalising the Feynman variational method to many-body Hamiltonians (e.g. many electrons rather than one) or even grand canonical systems with a variable number of particles, by using the toolkit of many-body Green functions \cite{Giustino2017} via the analogies I have identified. I think that a next logical step would be to derive the variational method for arbitrary representations of the path integral, such as the phase-space representation for canonical ensembles, and coherent state path integrals for grand-canonical ensembles \cite{Altland_Simons_2010}. This way, we may be able to go beyond the parabolic-band approximation for the electron (or other particles) to general electron band-structures. This may remove the discrepancy found between the predicted polaron binding energy at higher coupling using the derived variational Holstein method compared to those calculated using Diagrammatic Monte Carlo (as shown in Fig. 6.1).
% \newline 

% The `ultimate' vision for this project would be to have a highly accurate, potentially ab-initio, variational method for predicting both thermodynamic and dynamical properties of real complex materials and to apply it across whole classes of materials to discover trends that would aid material design and discovery. The attractiveness of this method over other methods, such as Monte-Carlo methods like Path Integral or Diagrammatic Monte Carlo, or purely self-consistent methods such as GW-theory or DMFT, is that this variational path integral method provides an efficient framework for generalising to many different quantum systems by making use of powerful toolkits from Quantum Field Theory, Functional variation and even potentially Many-Body Green Functions. Likewise, it has controlled errors, unlike Monte-Carlo, and has a well defined variational bound due to the convex nature of the path integrand, allowing for computationally tractable optimisation. 
% \newline

% However, I would be amiss to neglect a glaring limitation and potential problem with the variational method. In my analysis of the multiple fictitious particle trial action, I showed in Fig. 6.13 that the higher frequency dynamics of the system can be extremely sensitive to the derived optimal ground-state of the thermodynamic system. This was also investigated by Dries Sels \cite{Dries2016}. The cause of this discrepancy is due to the limited \emph{physics} of the trial system. Even the most general non-local quadratic action (Eqn. (\ref{eqn:generaltrial})) is still quadratic in its variables. In fact, a quadratic action indicates a lack of true particle-particle interactions in the theory as quadratic actions are typically associated with free non-interacting theories; the self-energy term here merely acts as a temporally non-local external potential. The problem can be made clearer by analogy to the Luttinger-Ward functional. As I discussed, the optimal trial action in the variational method amounts to an approximate truncation of the diagrams that compose a Luttinger-Ward-like functional (the sum of all closed, bold two-particle irreducible diagrams). I am not sure which diagrams precisely that the optimal Feynman variational method keeps, yet I aim to understand this for my thesis. Conceptually, I think it would be similar to the diagrams kept in cumulant Green function methods, so the variational method includes an infinite number of diagrams likely to higher and higher order in phonons interacting with the electron, but not all possible diagrams. Nonetheless, we still need to keep the trial action quadratic as the resultant path integral needs to be soluble for the variational method to have any merit. I mentioned previously that the Feynman variational method can be improved by including the second-order correction in the Feynman-Jensen inequality. I think this would then include another infinite set of diagrams which were previously absent, improving the overall approximation. We could continue in this way to improve the approximated Luttinger-Ward-like functional, adding more infinite sets of electron-phonon diagrams, where the optimal electron Green function is to be solved self-consistently. Obviously, the computational complexity of this method would be very limiting, which bring me to my next idea; machine learning.
% \newline

% Ultimately, what the Feynman-Jensen inequality does is map a complicated, intractable probability distribution to a simpler, optimal Gaussian distribution via a change of variable - albeit in an infinite-dimensional function space. This is a form of variational inference \cite{Neal1992}. The free energy inequality then has a \emph{functional} form of the Kullback-Leibler divergence often used in machine learning, which measures the statistical difference between the two continuous functional distributions. In the Feynman variational method, the Kullback-Leibler divergence is truncated to include only the first cumulant in the difference between the two distributions. A natural extension to higher order cumulants points to the use of \emph{normalising flows} \cite{Kobyzev2021}. So, instead of trying to solve the optimal Luttinger-Ward-like functional self-consistently, we could replace it with a series of deep neural-networks which are trained to minimise the Kullback–Leibler divergence between the trial model's likelihood and the true target distribution. The training data can be obtained by sampling directly from the target distribution as done for Path Integral Monte Carlo. A nice connection is that in the limit to a \emph{continuous} normalising flow, optimal-control methods may be used which can themselves be formulated in terms of Wiener stochastic path integrals and the Feynman-Kac formula. I do not know if I would have time to develop and code this machine learning method during my PhD or even if it have any practical merit, but I would at least like to include some theory behind it towards the end of my thesis.

Path integration is a powerful tool for finding accurate approximate solutions to the free energy of the polaron model and for describing the response of the polaron. The main result of my project so far has been the extension of the path integral approach to the polaron to explicitly include multiple phonon modes, and using the previously established techniques to make predictions of the complex conductivity. I will discuss here my interpretation of the results of my new multiple phonon model and the comparisons to Hellwarth's effective frequency model. Additionally, I will discuss the result of the application of my model to terahertz conductivity measurements of MAPbI$_3$, as well as the result of modelling anisotropy in materials in two recent papers. 

\subsection{Predictions from FHIP and DSG for the complex mobility}

The first thing to discuss is what one might expect to get from including multiple phonon modes given the result of a single mode. In~\cite{feynman_mobility_1962} and~\cite{devreese_optical_1972}, the predicted imaginary component of the memory function $\text{Im}\chi(\nu)$ and real component of the complex conductivity $\text{Re}\sigma(\nu)$ always included an initial peak starting sharply at the phonon mode frequency when at zero temperature. This is the only peak present for lower values of the electron-phonon coupling with $\alpha < 4.5$. It is only at couplings stronger than this that extra oscillatory peaks arise at multiples of the phonon mode frequency multiplied by the value of the variational parameter $\omega_{LO} \times v$. 

This new frequency, $\omega_{LO} \times v$, can be thought of as the polaron frequency, and so these extra peaks would be identified as polaron excitations. The first peak also develops a bit of an initial shoulder followed by a tall, sharp peak. In DSG they identified these peaks as an initial one-phonon peak at $\Omega = \omega_{LO}$ at lower $\alpha$s. As $\alpha$ increases, extra oscillator strength is added due to transitions to final states where lattice adaption to excited electronic configurations (Relaxed Excited States or RES) has occurred. A further increase in coupling produces a separation of the one-phonon and RES states. Eventually a side-band structure forms, which leads to a broad Frank-Condon (FC) peak that is a superposition of multiphonon sidebands and is less pronounced than the RES peak. 

At strong coupling the conductivity becomes very `structured' and it is hard to identify clear features. The extra `structure' is likely due to the breakdown of the Fr\"ohlich model in the strong-coupling limit, where the linewidth of the FC peak becomes smaller than $\omega_{LO}$. This violates the lifetime of $1 / \omega_{LO}$ derived from uncertainty relations and signals the breakdown of the Fr\"ohlich model at strong coupling due to the continuum approximation; the breakdown is not due to the FHIP approximation. 

Devreese has used an operator based theory \cite{Devreese2001} for the strong coupling limit that agrees well with Diagrammatic Monte Carlo data. This would explain the apparent breakdown of the conductivity in the $\alpha = 9$ plots. 

\subsection{Comparison of the multiple phonon and effective frequency mobilities}

I found that regardless of coupling strength, all peaks broaden, flatten and blue-shift to high frequencies as the temperature increases. Eventually, for temperatures greater than $100$ K, we recover a Drude-like response with all the oscillations dampened away. 

Including multiple phonon modes would superimposed multiple peaks in the memory function $\chi(\nu)$ (Eq. (\ref{eqn:multi_memory})), each starting at each of the respective phonon mode frequencies. The magnitude of these peaks would depend on the relative contributions of the modes to the overall electron-phonon coupling. The coupling would be proportional to the infrared activity of each respective mode. 

Comparing the memory function and the complex conductivity in DSG, we expect that the complex conductivity would possess similar structure to the memory function, with peaks starting at each respective phonon mode frequency. However, the overall structure is not the result of a straight forward superposition and instead involves the reciprocal of the sum over phonon modes as seen from Eq. (\ref{eqn:freq_dep_mobility}). Since the majority of real materials have Fr\"ohlich $\alpha$ values less than $4.5$, we would not expect to see any additional oscillations at higher frequencies corresponding to polaron excitations. Any extra peaks would be due to some combination of the many one-phonon peaks from each mode. 

From Figure $9$, we see that the multiple phonon mode mobility $9a$ has a very similar form to the single effective mode mobility $9b$, with one primarily broad peak starting around the Hellwarth and Biaggio effective frequency $2.03$ THz. This broad peak in the multiple phonon mobility is likely a result of the combined contributions from the $2.080$ THz, $2.249$ THz and $2.438$ THz modes which, from Table 1, we can see have the first, second and fourth largest contributions to the decomposed $\alpha_j$ parameter. Due to the combination of these three modes to form the broad peak, the multiple phonon mobility shows extra detail in the broad peak. The extra detail appears as a small initial oscillation at the start of the peak as seen in Figure $10$. 

Below $2$ THz in Figure $9$, the multiple phonon mobility possesses three main extra peaks that do not appear in the effective frequency mobility. These peaks are sharper than the broad peak and begin around $0.60$ THz, $1.00$ THz and $1.50$ THz. From Table 1 we see that the $0.574$ THz mode has the third largest contribution to the decomposed alpha parameter $\alpha_j$ and is likely the source of the first sharp peak around $0.60$ THz. The cluster of modes ranging from $0.801$ THz to $1.019$ THz are probably the source of the peak that starts around $1.00$ THz. Finally, the $1.567$ THz mode has a comparatively intermediate contribution to $\alpha_j$ and is likely the source of the small peak around $1.50$ THz, which appears more like a shoulder off of the back of the $1.00$ THz peak.

\subsection{Comparison of the multiple phonon mobility with THz photo-conductivity data}

From the photo-conductivity measurements, the two strong resonances around $1.25$ THz and $2.25$ THz are attributed to two groups of optical phonon modes associated with the bending and stretching of the Pb-I bond in MAPbI$_3$. In~\cite{zheng_multipulse_2021} we conclude that the origin of the non-Drude-like spectral response cannot be explained primarily by non-polaronic free carriers alone, and will likely require further investigation with non-equilibrium response theories. Nonetheless, the dominant effect of reducing the conductivity is attributed to the excitation of the electronic states as well as an interplay between the polaron and surrounding lattice that result in the peaks in the conductivity. The scenario described is one where the underlying phonons, that are strongly coupled to the charge-carriers and are a part of polaron formation, are heated by the carriers as the carriers cool. This occurs before the heat can be dissipated from these strongly coupled phonon modes scattering into non-coupled phonons (that are not involved in the polaron formation). 

Comparison with the multiple mode conductivity in Figure (\ref{fig:athermal_thz}) leads to reasonable qualitative agreements where the main resonances in the predicted response seem to align with the observed response. The $2.25$ THz peak is far broader in the theory, but it should be noted that the form the predicted response takes is sensitive to the modes provided, so that alternative measured/predicted values of phonon mode frequencies and infrared activities would alter the predicted spectral response. 

The best agreement with the measurements occurs when the effective temperature of the system is near to zero. This suggests that the sub-picosecond time scale of the THz probe measurement corresponds to a pre-thermal equilibrium mobility regime. It is not entirely clear how this can be established qualitatively and will the subject of further investigations. 

The multiple phonon mode model has the capability of involving more than just two $v$ and $w$ variational parameters, which would correspond to have more fictitious harmonic oscillators coupled to the charge-carrier in the model system. It would be interesting to investigate how including more harmonic oscillators changes the predicted response. It may be that matching the number of fictitious oscillators to the number of strongly coupled phonon modes would make more accurate predictions. However, I have not been able to investigate this yet due to issues with computational convergence of the variational expression in Eq. (\ref{eqn:multi_feynman_jensen}).

\subsection{Comparison of the variational and full perturbative approaches for modelling anisotropy}

From~\cite{guster_frohlich_2021} and shown in Figure (\ref{fig:anisotropy}), we find reasonable agreement in the weak-coupling regime between the proper perturbative treatment of anisotropy in the Fr\"ohlich model and the naive inclusion of anisotropy in the athermal Feynman path integral model. The best agreement is clearly for the inherently isotropic materials, with poorer agreement for the more anisotropic materials, although the difference is minor. For the anisotropic materials, the Feynman approach seems to over-estimate the in-plane effective masses compared to the full perturbative approach. This is likely due to the improper treatment of the anisotropic mass in the variational principle when determining the variational parameters that correspond to the in- and out-of-plane directions. This would explain the underestimate of the ground-state energy by the Feynman approach too, although it should be noted that the Feynman approach will always be expected to predict a lower ground-state energy than the perturbative approach. 

I recently found a 1982 paper by Peeters and Devreese \cite{Devreese1982} where they extend Feynman's polaron theory to account for anisotropy of the effective electron-phonon interaction under the influence of a perturbing magnetic field. The extended free energy variational principle in \cite{Devreese1982} would give the foundation for a proper treatment of anisotropic in the path integral approach to obtaining polaron effective masses. When discussing comparisons between the Feynman and perturbative approaches, we found that defining a polaron radius can be quite ambiguous. It would be useful to find a formal definition for the polaron radius to help with comparison between difference theoretical approaches. One way to do this may be to look at the maxima of the dynamic structure factor which may help define an effective polaron radius. In terms of the path integral model, the dynamic stucture factor is found by evaluating $\langle \exp(i \vb{k} \cdot [\vb{r}_{el}(t) - \vb{r}_{el}(0)]) \rangle_{S_0}$ Eq. ((\ref{eqn:S})) as indicated in \cite{Devreese2001} and \cite{Devreesetwo}.