\clearpage{}

\pagestyle{body}

\chapter{Extending the Variational Method to the Fr\"ohlich Model}
\label{chap:third}

\thesisepisrcyear{What is the meaning of life?}{John Doe}{Thoughts}{1971}

\chapterintrobox{This is the introduction paragraph.}

\section{Multiple Phonon Modes}
\label{sec:chap-third-first}

\lettrine{I}n simple polar materials with two atoms in the basis, the single triply-degenerate optical phonon branch is split by dielectric coupling into the singly-degenerate longitudinal-optical mode and double-generate transverse-optical modes. Only the longitudinal-optical mode is infrared active, and contributes to the Fr\"ohlich dielectric electron-phonon interaction. 

The infrared activity of this mode is driving the formation of the polaron, and similarly in a more complex material the range of infrared active modes all contribute to the polaron stabilisation. This relationship is, however, slightly obscured by the algebra in Eq. (\ref{eqn:frohlich_alpha}), and instead this electron-phonon coupling seems to emerge from bulk phenomenological quantities. The Pekar factor, $\frac{1}{\epsilon_{\infty}}-\frac{1}{\epsilon_{0}}$ being particularly opaque. Rearranging the Pekar factor as

\begin{equation}
    \left( \frac{1}{\epsilon_{\infty}} - \frac{1}{\epsilon_{0}} \right) = \frac{\epsilon^{ionic}}{\epsilon_{\infty}\epsilon_{0}},
    \label{eqn:pekar}
\end{equation}

we can now see that the Fr\"ohlich $\alpha$ is proportional to the ionic dielectric contribution $\epsilon^{\text{ionic}}_j$, as would be expected from appreciating that this is the driving force for polaron formation. 

The relative static dielectric constant is composed out of the high-frequency optical component (from the response of the electronic structure), and then the THz scale vibrational motion of the ions, $\epsilon_{0}=\epsilon_{\infty}+\epsilon_{j}$. This vibrational contribution is typically calculated (\cite{gonze_dynamical_1997}) by summing the infrared activity of the individual harmonic modes as Lorentz oscillators. This infrared activity can be obtained by projecting the Born effective charges along the dynamic matrix (harmonic phonon) eigenvectors. The overall dielectric function across the phonon frequency range can be written as 

\begin{equation}
    \begin{gathered}
    \epsilon^0_{\alpha \beta}(\omega) = \epsilon^{\infty}_{\alpha \beta} + \sum_{j}^{modes} \epsilon^{ionic}_{\alpha \beta j}(\omega)
    = \epsilon^{\infty}_{\alpha \beta} + \frac{4\pi e^2}{\Omega_0} \sum_{j \nu\mu}\frac{ \sum_{\alpha'}Z^{*\mu}_{\alpha\alpha'} u_{\mu j}^{\alpha'}  \sum_{\beta'}Z^{*\nu}_{\beta\beta'}  u_{\nu j}^{\beta'}}{\left(\omega_{j}^2 - \omega^2   \right)}
    \end{gathered}
\end{equation}

where $e$ is the electron charge, $\Omega_0$ the unit cell volume, $Z^{*\nu}_{\alpha \beta}$ is the Born effective charge tensor at atom $\nu$, $u^\alpha_{\mu j}$ is the dynamic matrix eigenvector at atom $\mu$ for the $j$th phonon branch, $\omega_{j}$ is the dispersionless LO phonon frequency for the $j$th phonon branch and $\omega$ is the reduced frequency. 

Considering the isotropic case (and therefore picking up a factor of $\frac{1}{3}$ for the averaged interaction with a dipole), and expressing the static (zero-frequency) dielectric contribution, in terms of the infrared activity of a mode, $\epsilon^{\text{ionic}}_{j}$ is 

\begin{equation}
\begin{split}
    \epsilon^{ionic}(\hat{k}) &= \sum_j^{modes} \epsilon^{ionic}_{j}(\hat{k})
    = \frac{4\pi e^2}{\Omega_0} \sum_{j}^{modes} \frac{\left(\sum_{\nu\alpha\beta} k^\alpha Z^{*\nu}_{\alpha \beta} u^\beta_{\nu j}\right)^2}{k^2 \omega_{j}^2}.
\end{split}
\end{equation}

This provides a clear route to defining $\alpha_j$ for individual phonon branches, with the simple constitutive relationship that $\alpha=\sum_j \alpha_j$:

\begin{equation}
    \begin{split}
    |V_\mathbf{k}|^2 &= \sum_j^{modes} \frac{4\pi \hslash (\hslash \omega_{j})^{3/2}}{\sqrt{2 m_b} \Omega_0 k^2} \alpha_j(\hat{k})
    = \frac{2\pi \hslash}{\Omega_0 k^2} \sum_j^{modes}\frac{\omega_{j} \epsilon^{ionic}_j(\hat{k})}{\epsilon_\infty(\hat{k}) \epsilon_0(\hat{k})}
    \end{split}
\end{equation}

where

\begin{equation}
\alpha_j = \frac{1}{4\pi\epsilon_0}  \frac{\epsilon_j}{\epsilon_{\infty}\epsilon_{0}} \frac{e^2}{\hslash} \left( \frac{m_b}{2\hslash\omega_j} \right)^{\frac{1}{2}}.
    \label{eqn:alphai}
\end{equation}

This concept of decomposing $\alpha$ into constituent pieces associated with individual phonon modes is implicit in the effective mode scheme of Hellwarth and Biaggio (Eqs. (\ref{eqn:hellwarth_scheme_s}) to (\ref{eqn:hellwarth_scheme_f})), and has also been used by Verdi,~\cite{verbist_extended_1992} and~\cite{devreese_many-body_2010}.

\subsubsection{Multiple phonon mode path integral}

\cite{verbist_extended_1992} proposed an extended Fr\"ohlich model Hamiltonian in Eq. (\ref{eqn:frohlich_hamiltonian}) with a sum over multiple ($m$) phonon branches,

\begin{equation}
    \hat{H} = \frac{p^2}{2m_b} + \sum_{\mathbf{k}, j} \hslash \, \omega_{j} \, a_{\mathbf{k}, j}^\dagger a_{\mathbf{k}, j}
    + \sum_{\mathbf{k}, j} ( V_{\mathbf{k}, j} \, a_{\mathbf{k}, j} \, e^{i\mathbf{k} \cdot \mathbf{r}} + V_{\mathbf{k}, j}^* \, a_{\mathbf{k}, j}^\dagger \, e^{-i\mathbf{k} \cdot \mathbf{r}}) .
\label{eqn:multifrohlich}
\end{equation}

Here the index $j$ indicates the $j$th phonon branch. The interaction coefficient is given by,

\begin{equation}
    V_{\mathbf{k}, j} = i\frac{2 \hslash \omega_j}{\mathbf{k}} \left(\sqrt{\frac{\hslash}{2 m_b \omega_j}} \frac{\alpha_j \pi}{\Omega_0} \right)^{1/2},
\end{equation}

with $\alpha_j$ as in Eq. (\ref{eqn:alphai}). From this Hamiltonian we get the following extended model action to use within the Feynman variational theory,

\begin{equation}
        S_j[\mathbf{r}(\tau)] =
        \frac{m_b}{2}\int^{\beta_j}_0 d\tau \left(\frac{d\mathbf{r}(\tau)}{d\tau}\right)^2 -
        \frac{\hslash^{3/2}}{2\sqrt{2 m_b}} \alpha_j \omega_{j}^{3/2} \int^{\beta_j}_0 d\tau \int^{\beta_j}_0 d\sigma \frac{G_j(|\tau - \sigma|)}{|\mathbf{r}(\tau) - \mathbf{r}(\sigma)|} .
\label{eqn:multiaction}
\end{equation}

where I introduce the reduced thermodynamic temperature for the $j$th phonon branch $\beta_j = \hslash \omega_j / (k_B T)$. $G_j(x)$ is the phonon Green's function for a phonon with frequency $\omega_j$, 

\begin{equation}
    G_j(x) = \frac{\cosh{(\beta_j/2-x)}}{\sinh{(\beta_j/2)}} .
\end{equation}

This form of action is consistent with Hellwarth and Biaggio deduction that multiple phonon branches results in the interaction term simply becoming a sum over terms with phonon frequency $\omega_j$ and coupling constant $\alpha_j$ dependencies as shown in Eq. (\ref{eqn:hellwarth_multi_action}). 

I now choose a suitable trial action to use with the action in Eq. (\ref{eqn:multiaction}). One option is to use Feynman's original trial action with two variational parameters, $C$ and $w$, which physically represents a particle (the charge carrier) coupled harmonically to a single fictitious particle (the additional mass of the quasi-particle due to interaction with the phonon field) with a strength $C$ and a frequency $w$. 

Obviously the dynamics of this model cannot be more complex than can be arrived at with the original Feynman theory, though the direct variational optimisation may get a better fit by using Hellwarth's and Biaggio's effective phonon mode approximation (Eqs. (\ref{eqn:hellwarth_scheme_s}) to (\ref{eqn:hellwarth_scheme_f})).
In order to build a model capable of expressing richer dynamics, and more directly describe real materials with multiple phonon modes, I extend Feynman's trial action to represent a particle (the charge-carrier) coupled to $n$ massive fictitious particles. This results in $2\times n$ variational parameters (one for the coupling strength and coupling frequency of each fictitious particle, per phonon branch). 

I extend Feynman's trial action using equations (1.1), (3.11) and (3.16) by \cite{poulter_complete_1992} which are given by:

\begin{subequations}

    A non-local harmonic action:
    
    \begin{equation}
    \begin{gathered}
        S = \frac{m}{2} \int^t_0 d\tau\ \vb{\dot{r}}(\tau)^2 - \frac{1}{8} \sum^n_{p = 1} \kappa_p \Omega_p \int^t_0 d\tau \int^t_0 d\sigma \frac{\cos(\Omega_p [t/2 - \abs{\tau - \sigma}])}{\sin(\Omega_p t / 2)} \left( \vb{r}(\tau) - \vb{r}(\sigma) \right)^2 \\
        + \int^t_0 d\tau\ \vb{F}(\tau) \cdot \vb{r}(\tau) \qquad \text{(1.1)}
    \end{gathered}
    \end{equation}
    
    a term corresponding to Hellwarth and Biaggio's $C$ expression in Eq. (\ref{eqn:hellwarth_C})
    
    \begin{equation}
        \frac{\partial}{\partial \kappa_p} \ln G(t) = \frac{3}{2} \sum_{q=1}^n \frac{h_q}{\omega_q} \frac{1}{\omega_q^2 - \Omega_p^2} \left( \frac{1}{\omega_q} - \frac{t}{2} \cot\left(\frac{\omega_q t}{2}\right) \right) \qquad \text{(3.11)}
    \end{equation}
    
    where
    
    \begin{equation}
        \kappa_{p} = m \left(\omega_{p}^2 - \Omega_{p}^2 \right) \prod\limits_{\substack{q=1 \\ q\neq p}}^n \frac{\omega_{q}^2 - \Omega_{p}^2}{\Omega_{q}^2 - \Omega_{p}^2}, \qquad h_{p} = \frac{1}{m} \left( \omega_{p}^2 - \Omega_{p}^2 \right) \prod\limits_{\substack{q=1 \\ q\neq p}}^n \frac{\Omega_{q}^2 - \omega_{q}^2}{\omega_{q}^2 - \omega_{q}^2}
    \end{equation}
    
    and the path integral pre-factor $G(t)$ that corresponds to the trial partition function $Z_{S_0}(\beta) = \exp(-\beta F_{S_0})$ where $F_{S_0}(\beta) = A(\beta)$ is the trial free energy proportional to Hellwarth and Biaggio's $A$ expression in Eq. (\ref{eqn:hellwarth_A})
    
    \begin{equation}
        G(t) = \left(\frac{m}{2\pi i \hslash t}\right)^{3/2} \prod_{p=1}^n \left( \frac{\omega_p}{\Omega_p} \frac{\sin(\Omega_p t / 2)}{\sin(\omega_p t / 2)} \right)^3. \qquad \text{(3.16)}
    \end{equation}
    
\end{subequations}

I then perform a Wick-rotation and equate the total time-like variable elapsed ($t$ in~\cite{poulter_complete_1992}) with $t \rightarrow -i\hslash\beta$. I recognise the parameters $\omega$ and $\Omega$ in~\cite{poulter_complete_1992} with Feynman's $v$ and $w$ respectively. This gives us a polaron trial action extended to a set $n$ of variational parameters per phonon branch,

\begin{equation} \label{eqn:multi_trial_action}
    \begin{split}
        S_{0j}[\mathbf{r}(\tau)] &=
        \frac{m_b}{2}\int^{\beta_j}_0 d\tau \left(\frac{d\mathbf{r}(\tau)}{d\tau}\right)^2 \\
        &+ \frac{1}{8} \sum_{p = 1}^n \kappa_{p} w_{p} \int^{\beta_j}_0 d\tau \int^{\beta_j}_0 d\sigma \frac{\cosh{(w_{p}[\beta_j/2-|\tau-\sigma|])}}{\sinh{(w_{p}\beta_j/2)}}(\mathbf{r}(\tau) - \mathbf{r}(\sigma))^{2} .
    \end{split}
\end{equation}

Here $\kappa_{p}$ is the spring constant associated with the $p$th fictitious particle and $w_{p}$ is the corresponding frequency of oscillation. 

\subsubsection{Multiple phonon mode free energy}

I extend Hellwarth and Biaggio's $A$ (\ref{eqn:hellwarth_A}) and $C$ (\ref{eqn:hellwarth_C}) equations, 

\begin{subequations}
\begin{align}
    A_j = &\frac{3}{\beta_j m} \left[ \sum_{p = 1}^n \left( \log\left(\frac{v_{p} \sinh (w_{p} \beta_j / 2)}{w_{p} \sinh (v_{p} \beta_j / 2)}\right) \right) - \frac{1}{2} \log \left(2\pi\beta_j\right) \right] , \label{eqn:A} \\
    &C_j = \frac{3}{m} \sum_{p = 1}^n \sum_{q = 1}^n \frac{C_{pq}}{v_{q} w_{p}} \left( \coth \left( \frac{v_{q} \beta_j}{2} \right) - \frac{2}{v_{q} \beta_j} \right) . \label{eqn:C}
\end{align}
\end{subequations}

With, 

\begin{subequations}
    \begin{align}
        C_{pq} = \frac{w_{p}}{4} &\frac{\kappa_{p} h_{q}}{v_{q}^2 - w_{p}^2} ,\\
        \kappa_{p} = \left(v_{p}^2 - w_{p}^2 \right) &\prod\limits_{\substack{q=1 \\ q\neq p}}^n \frac{v_{q}^2 - w_{p}^2}{w_{q}^2 - w_{p}^2} ,\\
        h_{p} = \left( v_{p}^2 - w_{p}^2 \right) &\prod\limits_{\substack{q=1 \\ q\neq p}}^n \frac{w_{q}^2 - v_{q}^2}{v_{q}^2 - v_{q}^2} .
    \end{align}
\end{subequations}

$C_{pq}$ are the components of a generalised ($n \times n$) matrix version of Feynman's $C$ variational parameter. The cross (off-diagonal) terms give the coupling (interaction) between the fictitious particles. 

Now I note that the $R(x)$ function appearing in Eq. (3.8) in \cite{poulter_complete_1992} 

\begin{equation}
    R(x) = 2 \sum_{p = 1}^n \frac{h_p}{\omega_p^3} \frac{\sin(\omega_p x / 2) \sin(\omega_p [t - x] / 2)}{\sin(\omega_p t / 2)} + \frac{1}{mt} \left( 1 - m \sum_{p = 1}^n \frac{h_p}{\omega_p^2} \right) x (t - x)
\end{equation}

is a generalisation of the $D(x)$ expression in FHIP given in Eq. (\ref{eqn:D_FHIP}), which in more familiar notation and again Wick-rotated to the form given in Eq. (\ref{eqn:FD}) gives

\begin{equation}\label{eqn:multi_D}
\begin{gathered}
    D_j(x) = 2 \sum_{p=1}^n \frac{h_{p}}{v_{p}^3} \frac{\sinh{(v_{p} x/2)\sinh{(v_{p}[\beta_j-x]/2)}}}{\sinh(v_{p}\beta_j/2)}
    + \left( 1 - \sum_{p = 1}^n \frac{h_{p}}{v_{p}^2} \right) x \left(1 - \frac{x}{\beta_j}\right).
\end{gathered}
\end{equation}

When $n=1$ (a single ficticious particle) with $x \rightarrow -iu$ and $t \rightarrow -i\hslash\beta$, $D_j(x)$ is the same form as $D(u)$ fromEq. (\ref{eqn:FD}) from Feynman's polaron theory. 

Combining $D_j(x)$ in Eq. (\ref{eqn:multi_D}) and the multiple phonon action in Eq. (\ref{eqn:multiaction}), we arrive at a generalisation to Hellwarth and Biaggio's B expression, including multiple ($m$ with index $j$) phonon branches, and multiple ($2n$ with index $p$) variational parameters $v_{p}$ and $w_{p}$,

\begin{equation}
\begin{gathered}
    B_j = \frac{\alpha_j}{\sqrt{\pi}} \int_0^{\frac{\beta_j}{2}} d\tau \frac{\cosh (\beta_j / 2 - \tau)}{\sinh(\beta_j / 2)} \left[ D_j(\tau) \right]^{-\frac{1}{2}} .
\label{eqn:B}
\end{gathered}
\end{equation}

Summing the trial free energy $A_j$ in Eq. (\ref{eqn:A}), the trial-model interaction $B_j$ in Eq. (\ref{eqn:B}), and the trial action $C_j$ in Eq. (\ref{eqn:A}), we obtain a generalised variational inequality for the contribution to the free energy of the polaron from the $j$th phonon branch with phonon frequency $\omega_j$ and coupling constant $\alpha_j$, and $2n$ variational parameters $v_{p}$, $w_{p}$, 

\begin{equation} \label{eqn:multi_feynman_jensen}
    \begin{gathered}
         F(\beta) \leq -\sum_{j=1}^m \hslash \omega_j \left(A_j + B_j + C_j\right) \\ = \sum_{j=1}^m \hslash \omega_j \left\{ \frac{3}{\beta_j m} \left[ \sum_{p = 1}^n \left( \log\left(\frac{v_{p} \sinh (w_{p} \beta_j / 2)}{w_{p} \sinh (v_{p} \beta_j / 2)}\right) \right) - \frac{1}{2}\log \left(2\pi\beta_j\right) \right] \right. \\ \left. + \frac{3}{m}\sum_{p = 1}^n \sum_{q = 1}^n \frac{C_{pq}}{v_{q} w_{p}} \left( \coth \left(\frac{v_{q}\beta_j}{2}\right) - \frac{2}{v_{q}\beta_j}\right) \right. \\ 
        \left. + \frac{\alpha_j}{\sqrt{\pi}} \int_0^{\frac{\beta_j}{2}} d\tau \frac{\cosh(\beta_j/2 - \tau)}{\sinh(\beta_j/2)} \left[2 \sum_{p=1}^n \frac{h_{p}}{v_{p}^3} \frac{\sinh{(v_{p} \tau/2)\sinh{(v_{p}[\beta_j-\tau]/2)}}}{\sinh(v_{p}\beta_j/2)} \right. \right.\\
        \left. \left. + \left(1 - \sum_{p = 1}^n \frac{h_{p}}{v_{p}^2}\right) \tau \left(1 - \frac{\tau}{\beta_j}\right) \right]^{-\frac{1}{2}} \right\} .
    \end{gathered}
\end{equation}

Here I have taken care to write out the expression explicitly, rather than use ``polaron'' units as used in the literature. Despite that each of the phonon branches $j$ are independent, the RHS of Eq. (\ref{eqn:multi_feynman_jensen}) is minimised for total summation over each phonon branch to give the upper-bound for the total model free energy $F$. 

I obtain vectors of length $n$ for the variational parameters $v_{p}$ and $w_{p}$ that correspond to these minima, which will be used in evaluating the polaron mobility. When we consider only one phonon branch ($m = 1$) and only two variational parameters ($n = 1$) this simplifies to Hellwarth and Biaggio's form of \=Osaka's free energy in Eq. (\ref{eqn:hellwarth_energy}). Feynman's original athermal version can then be obtained by taking the zero temperature limit ($\beta \rightarrow \infty$).

\subsubsection{Multiple phonon mode mobility}

To generalise the frequency-dependent mobility in Eq. (\ref{eqn:freq_dep_mobility}) I follow the same procedure as FHIP, but use our generalised polaron action $S$ in Eq. (\ref{eqn:multiaction}) and trial action $S_0$ in Eq. (\ref{eqn:multi_trial_action}). The result is a memory function akin to Eq. (\ref{eqn:fhip_chi}) that is inclusive of multiple ($m$) phonon branches $j$ and multiple ($2n$) variational parameters $v_{p}$ and $w_{p}$,

\begin{equation} \label{eqn:multi_memory}
    \begin{gathered}
        \chi(\Omega) = \sum_{j=1}^m \frac{2\alpha_j}{3\sqrt{\pi}} \int_0^{\infty} dt\ \left[1 - e^{i\Omega t / \omega_j}\right] \textrm{Im} S_j(t)
    \end{gathered}
\end{equation}

where

\begin{equation}
    S_j(\Omega) = \frac{\cos \left(t - i\beta_j/2\right)}{\sinh (\beta_j/2)} [D_j(t)]^{-3/2}
\end{equation}

where $D(t)$ is just $D_j(it)$ from Eq. (\ref{eqn:multi_D}) rotated back to real-time to give a generalised version of $D(u)$ in Eq. (\ref{eqn:D_FHIP}) from FHIP,

\begin{equation}
    \begin{gathered}
        D_j(t) \equiv D_j(it) = 2\sum_{p=1}^n \frac{h_p}{v_{p}^3} \frac{\sin(v_{p} t/2) \sin(v_{p}[t-i\beta_j]/2)}{\sinh(v_{p}\beta_j/2)} \\
        -i \left(1-\sum_{p=1}^n\frac{h_{p}}{v_{p}^2}\right) t \left(1 - \frac{t}{i\beta_j}\right).
    \end{gathered}
\end{equation}

The new multiple-phonon frequency-dependent mobility $\mu(\Omega)$ is then obtained from the real and imaginary parts of the generalised $\chi$ using Eq. (\ref{eqn:freq_dep_mobility}). The frequency-dependent mobility $\mu(\Omega)$ is obtained from the impedance using

\begin{equation}\label{eqn:freq_dep_mobility}
\begin{gathered}
    \mu(\Omega)^{-1} = \frac{m_b}{e} \sum_j^m \omega_j \textrm{Re}\left\{z_j(\Omega)\right\}
    = \frac{m_b}{e} \sum_j^m \omega_j \frac{\Omega^4 - 2\ \Omega^2\  \textrm{Re}\chi_j(\Omega) + |\chi_j(\Omega)|^2}{\Omega\ \textrm{Im}\chi_j(\Omega)}
    \end{gathered}
\end{equation}

where $\chi_j(\Omega)$ is just the $j$th component of $\chi(\Omega)$. The limit that the frequency $\Omega \rightarrow 0$ gives the FHIP dc-mobility extended to multiple phonon modes,

\begin{equation}
    \mu^{-1}_{dc} = \frac{m_b}{e}\lim_{\Omega \rightarrow 0} \sum_{j=1}^m \omega_j \frac{\textrm{Im}\chi_j(\Omega)}{\Omega}
\end{equation}

since $\textrm{Re}\chi(\Omega = 0) = 0$.

\subsubsection{Numerical integration of the memory function $\chi(\Omega)$}

In~\cite{feynman_mobility_1962}, and in~\cite{devreese_optical_1972}, they first convert the integral in the complex memory function $\chi(\Omega)$ (Eq. (\ref{eqn:fhip_chi})) to a contour integral (Eq. (\ref{eqn:contour_imX})) (I provide more details in Appendix A). The contour integral is then evaluated by expanding it as an infinite power series in terms of $K$, Bessel functions with imaginary argument. 

I investigated three methods for evaluating the real and imaginary components of the complex memory function: $(1)$ direct Gauss-Kronrod quadrature integration of the memory function in Eqs. (\ref{eqn:fhip_chi}) and (\ref{eqn:multi_memory}); $(2)$ Gauss-Kronrod integration of the contour integrals (Eq. (\ref{eqn:contour_imX}) and a corresponding contour integral of the multiple phonon memory function (\ref{eqn:multi_memory})); and $(3)$ using the power series expansions of Bessel functions. I found that the contour integrals were doubly oscillatory and became unstable towards high frequencies ($\gtrapprox 30$ multiples of the phonon mode frequency $\omega_{LO}$). The power series expansions required arbitrary-precision floating-point numbers to avoid diverging at low temperatures or high frequencies. Therefore, method $(1)$, direct Gauss-Kronrod quadrature integration of the original integrals, was the most numerically stable and my chosen method for evaluating the complex memory functions.

\subsection{Na\"ive effective-mass anisotropy}

In the degenerate anisotropic uni-axial case, I propose to na\"ively incorporate the anisotropy into the Feynman approach (which is one-dimensional due to the underlying isotropy of the Fr\"ohlich Hamiltonian) by treating the two directions independently with effective masses $m_\perp$ and $m_z$.

I then use the variational principle separately in each direction to find the variational parameters $v_{\perp/z}$ and $w_{\perp/z}$ that give the lowest upper-bound to the ground-state energy $E_{\perp/z}$ for each direction. 

The variational parameters can then be used to obtain the effective polaron masses $m^{*F}_{P,\perp}$ and $m^{*F}_{P, z}$ using Eq. (\ref{eqn:mass_feynman}) and polaron radii $r_{P\perp}$ and $r_{Pz}$ using Eq. (\ref{eqn:pol_size_schultz}). 

To make comparisons with the isotropic case, I define an effective ground-state energy by taking the arithmetic mean of the uniaxial components of the ground-state energy,

\begin{equation}
    E = \frac{2 E_\perp + E_z}{3}.
\end{equation}

Similarly, I define an effective radius of the anisotropic polaron by finding a radius of sphere with the same volume of the ellipsoidal anisotropic polaron. This means taking the geometric mean of the uni-axial components of the polaron radius, 

\begin{equation}
    r_P = \left(r^2_{P\perp} r_{Pz}\right)^{1/3}.
\end{equation}

The two averaging methods are justified as they are the only methods found to give a ground-state energy and polaron radius consistent with those evaluated by the original model,~\cite{feynman_slow_1955}, when applied to an isotropic material.