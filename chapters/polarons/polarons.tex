\chapter{Polarons}
\label{chap:fourth}

\section{General Polaron}

The Hamiltonian for a general polaron model \cite{Alexandrov2009} can be written in second-quantisation form and momentum-basis as,
\begin{equation}
    \begin{aligned}
        H &= \sum_{\vb{k}} \epsilon_{\vb{k}} c^\dagger_{\vb{k}} c_{\vb{k}} + \sum_{\vb{q}} \hbar \omega_{\vb{q}} b^\dagger_{\vb{q}} b_{\vb{q}} + \sum_{\vb{k},\vb{q}} V_{\vb{k}, \vb{q}} c^\dagger_{\vb{k}+\vb{q}} c_{\vb{k}} (b^\dagger_{\vb{-q}} + b_{\vb{q}})
    \end{aligned}
\end{equation} 
where $\epsilon_{\vb{k}}$ is the electron band energy for momentum $\vb{k}$, $c^\dagger_{\vb{k}}$ and $c_{\vb{k}}$ are the electron creation and annihilation operators for an electron with momentum $\vb{k}$, $\omega_{\vb{q}}$ is the phonon frequency for momentum $\vb{q}$, $b^\gamma_{\vb{q}}$ and $b_{\vb{q}}$ are the phonon creation and annihilation operators for a phonon with momentum $\vb{q}$, $V_{\vb{k}, \vb{q}}$ is the electron-phonon coupling matrix which describes the strength of the interaction.
\newline

To apply the Feynman-Jensen variational approximation \cite{Feynman1955}, we simplify this general polaron Hamiltonian to the simpler case of a single electron or hole (conduction or valence) interacting with dispersionless optical phonons of frequency $\omega_0$,
\begin{equation}
    H = \frac{\vb{p}^2_{el}}{2m_b} + \hbar\omega_{0} \sum_{\vb{q}} 
    b^{\dagger}_{\vb{q}}b_{\vb{q}} + \sum_{\vb{q}} V_{\vb{q}} \rho_{\vb{q}} \left( b^\dagger_{-\vb{q}} + b_{\vb{q}} \right),
\end{equation}
where $\vb{p}_{el}$ is the momentum operator of the electron/hole, $m_b$ is the effective band-mass which for the Fr\"ohlich model \cite{frohlich} is obtained from first-principle calculations and for the Holstein model \cite{HolsteinI1959, HolsteinII1959} is obtained via a parabolic approximation of the tight-binding electron dispersion. $\rho_{\vb{q}}$ is the density operator, in second quantisation this is $\rho_{\vb{q}} = 
\sum_{\vb{k}} c^\dagger_{\vb{k}+\vb{q}} c_{\vb{k}}$ whereas in first quantisation it is $\rho_{\vb{q}} = e^{i \vb{q} \cdot \vb{r}_{el}}$.

\section{Continuum Polarons: The Fr\"ohlich Model}

In the Fr\"ohlich model \cite{frohlich}, we assume the system is a continuum so the electronic and phonon wave numbers are continuous and unbounded $-\infty < k, q < \infty$. The electron dispersion is,
\begin{equation}
    \epsilon_k^{(F)} = \frac{\hbar^2 k^2}{2 m_F} = \frac{p^2_{el}}{2 m_F},
\end{equation}
where the Fr\"ohlich effective band-mass is obtained from ab-initio calculations.

The electron-phonon interaction is long-range for the (dielectrically mediated) and in $n$-dimensions is given by \cite{Peeters1986},
\begin{equation}
    \begin{aligned}
        V^{(F)}_{\vb{q}} &= \frac{g_{F}(n)}{\sqrt{V q^{n-1}}}, \\
        &= \hbar \omega_0 \left(\frac{\alpha_F r_p}{V} \Gamma\left(\frac{n-1}{2}\right) \left( \frac{2 \sqrt{\pi}}{q} \right)^{n-1}\right)^{\frac{1}{2}},
    \end{aligned}
\end{equation}
where $g_{F}(n)$ and $V$ are respectively the Fr\"ohlich electron-phonon coupling strength and crystal volume in $n$ dimensions, $r_p = \sqrt{\hbar / (2 \omega_0 m_b)}$ is the characteristic polaron radius and $\alpha_F$ is the unitless electron-phonon coupling constant which can be calculated from material properties using,
\begin{equation}
    \alpha_F = \frac{e^2}{2 \hbar \omega_0 r_p} \frac{1}{4\pi \varepsilon_{\text{vac}}}\left(\frac{1}{\varepsilon_{\infty}} - \frac{1}{\varepsilon_0}\right),
\end{equation}
where $e$ is the electron charge and $\varepsilon_{\text{vac}}$, $\varepsilon_{\infty}$ and $\varepsilon_0$ are the vacuum, optical and static dielectric constants.

\section{Lattice Polarons: The Holstein Model}

In the Holstein model \cite{HolsteinI1959, HolsteinII1959}, the electronic and phononic wave numbers $k$ and $q$ assume $N$ discrete values within the first Brillouin zone $-\frac{\pi}{a} < k, q < \frac{\pi}{a}$, where $N$ is the number of sites and $a$ the lattice constant. The electron dispersion is,
\begin{equation}
    \epsilon^{(H)}_{\vb{k}} = -2 J \sum_{i=1}^n \cos(k_i) ,
\end{equation}
where $J$ is the nearest-neighbour electronic hopping amplitude. The electron-phonon interaction $V_{\vb{q}} \to V_0$ is just a constant given by,
\begin{equation}
    \begin{aligned}
        V^{(H)}_{0} &= \frac{g_{H}(n)}{\sqrt{N}} , \\
        &= \sqrt{\frac{2 J n \hbar \omega_0 \alpha_H}{N}} ,
    \end{aligned}
\end{equation}
where $g_{H}(n)$ is the Holstein electron-phonon coupling strength in $n$ dimensions and is parameterised by the \emph{unitless} coupling constant $\alpha_H$. As $J$ is the main energy scale in this model, we need to also define another unitless parameter $\gamma \equiv \hbar \omega_0 / J$ which is the ratio of the phonon energy and electron hopping amplitude and called the adiabaticity.
\newline

To use the Feynman-Jensen variational approximation, and in analogy to the Fr\"ohlich model, we approximate the electronic wave number to be small $k \ll 1$ and approximate the electron dispersion by expanding around $k = 0$ for a given orthogonal direction in k-space. Using the identity $\cos(\theta) \approx 1 - \theta^2 / 2 $ for $\theta \ll 1$, then by analogy with the free electron dispersion relationship $\epsilon_k = p^2 / 2 m_e = \hbar^2 k^2 / 2 m_e$ we have a tight-binding effective band-mass for the Holstein model $m_H$ defined as,
\begin{equation}
    m_H = \frac{\hbar^2}{2 J a^2} ,
\end{equation}
and an approximate electronic dispersion
\begin{equation}
    \epsilon^{(H)}_{\vb{k}} \approx \frac{p^2_{el}}{2 m_H}.
\end{equation}
Since we have an explicit expression for the effective band mass, we can also express the characteristic polaron length as $r^{(H)}_p = a / \sqrt{\gamma}$.

\section{The Fr\"ohlich Model}
\label{sec:chap-fourth-first}

\subsection{The Hamiltonian}
\label{subsec:chap-fourth-first}

\lettrine{T}{his} is the second chapter.
Morbi pharetra magna a lorem.
Cras sapien.
Duis porttitor vehicula urna.
Phasellus iaculis, mi vitae varius consequat, purus nibh sollicitudin mauris, quis aliquam felis dolor vel elit.
Quisque neque mi, bibendum non, tristique convallis, congue eu, quam.
Etiam vel felis.
Quisque ac ligula at orci pulvinar rutrum.

\Cref{fig:chap-second-algorithm} shows a pseudo-code algorithm.
\begin{figure}[!ht]
  \begin{algorithm}[H] % 'H' is used to remove floating attribute and leave it to 'figure' instead
  
    choose \(I\) and \(J\)\;
    \While{res \(>\) tol}{%
      \For{\(i\) \(\leftarrow\) \(-I\) \KwTo{} \(I\)}{%
        \For{\(j\) \(\leftarrow\) \(-J\) \KwTo{} \(J\)}{%
          compute \(A_{ij}\)\;
          }
        }
      compute res\;
    }
  \end{algorithm}
  \caption{Some pseudo-code algorithm}
  \label{fig:chap-second-algorithm}
\end{figure}
 