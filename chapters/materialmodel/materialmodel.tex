\chapter{Generalising the Material Path Integral Model}
\label{chap:3}

\thesisepisrcyear{
\textbf{Greensite} ``\ldots in a sense God did hand us the exact wave function \ldots - it's a Feynman path integral, and the only question is \ldots'' \\ \textbf{Feynman} ``\ldots how to do the integral, yeah!''\newline}{Jeff Greensite and Richard Feynman}{Variational Calculations in Quantum Field Theory}{1988}

\chapterintrobox{In this chapter, I focus on generalising the \emph{model} path integral to more complicated systems beyond the original Fr\"ohlich Hamiltonian. To extend the applicability of the variational method to more materials, I generalise the model path integral to multiple phonon modes in the Fr\"ohlich Hamiltonian; lattice polaron as described by the Holstein Hamiltonian; general electron-phonon interactions for a (possibly anisotropic) parabolic-band electron; and many polarons using coherent-state path integrals.}

\section{Multiple Phonon Modes}
\label{sec:3-1}

\lettrine{I}n simple polar materials with two atoms in the basis, the single triply-degenerate optical phonon branch is split by dielectric coupling into the singly-degenerate longitudinal-optical mode and double-generate transverse-optical modes. Only the longitudinal-optical mode is infrared active and contributes to the Fr\"ohlich dielectric electron-phonon interaction. 

The infrared activity of this mode drives the formation of the polaron. Similarly, in a more complex material, the range of infrared active modes contributes to the polaron stabilisation. This relationship is, however, slightly obscured by the algebra in Eq. (\ref{eqn:frohlich_alpha}), and instead, this electron-phonon coupling seems to emerge from bulk phenomenological quantities. The Pekar factor, $\frac{1}{\epsilon_{\infty}}-\frac{1}{\epsilon_{0}}$ being particularly opaque. Rearranging the Pekar factor as
\begin{equation}
    \left( \frac{1}{\epsilon_{\infty}} - \frac{1}{\epsilon_{0}} \right) = \frac{\epsilon^{ionic}}{\epsilon_{\infty}\epsilon_{0}},
    \label{eqn:pekar}
\end{equation}
we can now see that the Fr\"ohlich $\alpha$ is proportional to the ionic dielectric contribution $\epsilon^{\text{ionic}}_j$, as would be expected from appreciating that this is the driving force for polaron formation. 

\subsection{Multimodal Fr\"ohlich Hamiltonian}

The relative static dielectric constant is composed out of the high-frequency optical component (from the response of the electronic structure), and then the THz scale vibrational motion of the ions, $\epsilon_{0}=\epsilon_{\infty}+ \sum_j \epsilon_{j}$. This vibrational contribution is typically calculated by summing the infrared activity of the individual harmonic modes as Lorentz oscillators~\cite{gonze_dynamical_1997}. This infrared activity can be obtained by projecting the Born effective charges along the dynamic matrix (harmonic phonon) eigenvectors. The overall dielectric function across the phonon frequency range can be written as 
\begin{equation}
    \begin{gathered}
    \epsilon^0_{\alpha \beta}(\omega) = \epsilon^{\infty}_{\alpha \beta} + \sum_{j}^{modes} \epsilon^{ionic}_{\alpha \beta j}(\omega)
    = \epsilon^{\infty}_{\alpha \beta} + \frac{4\pi e^2}{\Omega_0} \sum_{j \nu\mu}\frac{ \sum_{\alpha'}Z^{*\mu}_{\alpha\alpha'} u_{\mu j}^{\alpha'}  \sum_{\beta'}Z^{*\nu}_{\beta\beta'}  u_{\nu j}^{\beta'}}{\left(\omega_{j}^2 - \omega^2   \right)}
    \end{gathered}
\end{equation}
where $e$ is the electron charge, $\Omega_0$ the unit cell volume, $Z^{*\nu}_{\alpha \beta}$ is the Born effective charge tensor at atom $\nu$, $u^\alpha_{\mu j}$ is the dynamic matrix eigenvector at atom $\mu$ for the $j$th phonon branch, $\omega_{j}$ is the dispersionless LO phonon frequency for the $j$th phonon branch and $\omega$ is the reduced frequency. 

Considering the isotropic case (and therefore picking up a factor of $\frac{1}{3}$ for the averaged interaction with a dipole), and expressing the static (zero-frequency) dielectric contribution, in terms of the infrared activity of a mode, $\epsilon^{\text{ionic}}_{j}$ is,
\begin{equation}
\begin{split}
    \epsilon^{ionic}(\hat{q}) &= \sum_j^{modes} \epsilon^{ionic}_{j}(\hat{q})
    = \frac{4\pi e^2}{\Omega_0} \sum_{j}^{modes} \frac{\left(\sum_{\nu\alpha\beta} q^\alpha Z^{*\nu}_{\alpha \beta} u^\beta_{\nu j}\right)^2}{q^2 \omega_{j}^2}.
\end{split}
\end{equation}
This provides a clear route to defining $\alpha_j$ for individual phonon branches, with the simple constitutive relationship that $\alpha=\sum_j \alpha_j$,
\begin{equation}
    \begin{split}
    |M_{\vb{q}}|^2 &= \sum_j^{modes} \frac{4\pi \hbar (\hbar \omega_{j})^{3/2}}{\sqrt{2 m_b} \Omega_0 q^2} \alpha_j(\hat{q})
    = \frac{2\pi \hbar}{\Omega_0 q^2} \sum_j^{modes}\frac{\omega_{j} \epsilon^{ionic}_j(\hat{q})}{\epsilon_\infty(\hat{q}) \epsilon_0(\hat{q})}
    \end{split}
\end{equation}
where,
\begin{empheq}{Multimode Fr\"ohlich Dimensionless Coupling}
\begin{equation}
\alpha_j = \frac{1}{4\pi\epsilon_0}  \frac{\epsilon_j}{\epsilon_{\infty}\epsilon_{0}} \frac{e^2}{\hbar} \left( \frac{m_b}{2\hbar\omega_j} \right)^{\frac{1}{2}}.
    \label{eqn:alphai}
\end{equation}
\end{empheq}
This concept of decomposing $\alpha$ into constituent pieces associated with individual phonon modes is implicit in the effective mode scheme of Hellwarth and Biaggio (Eqs. (\ref{eqn:hellwarth_scheme_s}) to (\ref{eqn:hellwarth_scheme_f})), and has also been used by Verdi,~\cite{verbist_extended_1992} and Devreese~\cite{devreese_many-body_2010}.

Verbist~\cite{verbist_extended_1992} proposed an extended Fr\"ohlich model Hamiltonian in Eq. (\ref{eqn-2122-20:frohlich_hamiltonian}) with a sum over multiple ($m$) phonon branches,
\begin{empheq}{Multimode Fr\"ohlich Polaron Hamiltonian}
\begin{equation}
    \hat{H} = \frac{p^2}{2m_b} + \sum_{\vb{q}, j} \hbar \, \omega_{j} \, b_{\vb{q}, j}^\dagger b_{\vb{q}, j}
    + \sum_{\vb{q}, j} ( M_{\vb{q}, j} \, b_{\vb{k}, j} \, e^{i\vb{q} \cdot \vb{r}} + M_{\vb{q}, j}^* \, b_{\vb{q}, j}^\dagger \, e^{-i\vb{q} \cdot \vb{r}}) .
\label{eqn:multifrohlich}
\end{equation}
\end{empheq}
Here, the index $j$ indicates the $j$th phonon branch. The interaction coefficient is given by,
\begin{equation}
    M_{\vb{q}, j} = i\frac{2 \hbar \omega_j}{\abs{\vb{q}}} \left(\sqrt{\frac{\hbar}{2 m_b \omega_j}} \frac{\alpha_j \pi}{\Omega_0} \right)^{1/2},
\end{equation}
with $\alpha_j$ as in Eq. (\ref{eqn:alphai}). 

\subsection{Multiple Phonon Mode Path Integral}
\label{subsec:3-1-1}

From the Multimodal Fr\"ohlich Hamiltonian, we get the following extended model action to use within the Feynman variational theory,
\begin{equation}
        S_j[\vb{r}(\tau)] =
        \frac{m_b}{2}\int^{\hbar\beta}_0 d\tau \left(\frac{d\vb{r}(\tau)}{d\tau}\right)^2 -
        \frac{\hbar^{3/2}}{2\sqrt{2 m_b}} \alpha_j \omega_{j}^{3/2} \int^{\hbar\beta}_0 d\tau \int^{\hbar\beta}_0 d\sigma \frac{D_{\omega_j}(\tau - \sigma)}{|\vb{r}(\tau) - \vb{r}(\sigma)|} .
\label{eqn:multiaction}
\end{equation}
where I introduce the reduced thermodynamic temperature for the $j$th phonon branch $\beta = 1 / (k_B T)$. $D_{\omega_j}(\tau)$ is the phonon Green's function for a phonon with frequency $\omega_j$, 
\begin{equation}
    D_{\omega_j}(\tau) = \frac{\cosh{(\omega_j(\tau - \hbar\beta/2))}}{\sinh{(\hbar\omega_j\beta/2)}}.
\end{equation}
This form of action is consistent with Hellwarth and Biaggio's deduction that multiple phonon branches result in the interaction term simply becoming a sum over terms with phonon frequency $\omega_j$ and coupling constant $\alpha_j$ dependencies as shown in Eq. (\ref{eqn:hellwarth_multi_action}). 

\section{Lattice Polarons}
\label{sec:3-2}

Hans De Raedt and Ad Lagendijk~\cite{de_raedt_numerical_1983, de_raedt_monte_1985} derived the discrete path integral for a lattice polaron, which was then further developed by Pavel Kornilovitch~\cite{kornilovitch_polaron_1997, kornilovitch_continuous-time_1998, kornilovitch_ground-state_1999, kornilovitch_giant_1999, kornilovitch_band_2000, kornilovitch_feynmans_2004, kornilovitch_path_2007}. Kornilovitch derived the continuous path integral limit of the lattice polaron and developed a Continuous-time Path Integral Monte Carlo method for calculating properties of small polarons, focusing on the Holstein model and a lattice version of the Fr\"ohlich model (which allows for long-range electron-phonon coupling). 

The path integral for the partition function of a system is a sum of all possible position and momentum paths in phase space. In the continuum large-polaron model, the sum over all momentum paths can be made as the momentum paths are continuous and unbounded, and the parabolic electronic dispersion is quadratic in momentum. Therefore, the corresponding momentum path integral is an evaluable Gaussian functional integral.

For the small lattice polaron, the electronic dispersion is that of a tight-binding model and is not quadratic. Likewise, the momentum and position paths are discrete, and the `path-integral' is an infinite summation rather than a functional integral. 

In the following section, I present the derivation of the discrete-time path integral for the Holstein model. I then take the continuous-time limit and show that the continuous-time path integral has the same form as the standard phase-space path integral but with the Holstein Hamiltonian.

\subsection{Discrete-Time Path Integral}
\label{subsec:3-2-1}

We begin with the Holstein Hamiltonian in a mixed representation:
\begin{equation}
    \begin{aligned}
        H &= H_{0} + H_{1} + H_{2} , \\
        H_{0} &= \frac{1}{2M} \sum_{i=1}^N p_i^2 , \\
        H_{1} &= \frac{M \omega_0^2}{2} \sum_{i=1}^N x_i^2 + g_H \sum_{n=1}^N x_i c^\dagger_i c_i , \\
        H_{2} &= -J \sum_{i=1}^N c^\dagger_i c_{i+1} + c_{i+1}^\dagger c_i ,
    \end{aligned}
\end{equation}
the phonons are expressed in their momenta $p_i$ and positions $x_i$ where $i$ labels the corresponding lattice site. $M$ is the mass of one lattice site (here, we assume all of them to have the same mass), and $\omega_0$ is the dispersionless phonon frequency where we assume to have only one mode (i.e. Einstein mode). The electron description remains in terms of the creation and annihilation operators $c^\dagger_i$, $c_i$ on a lattice-site $i$.

In deriving the path integral, a system's quantum statistical partition function may be obtained by inserting successive resolutions of identity within the definition of a quantum trace. In the limit of an infinite number of insertions, the Trotter-Suzuki expression \cite{trotter_product_1959, hatano_finding_2005} gives a direct equality between this discretised partition function $Z_M$ and the total partition function $Z$:
\begin{equation}
    \begin{aligned}
        Z &\equiv \Tr{e^{-\beta H}} = \lim_{M\to\infty} Z_M , \\
        Z_M &= \Tr{\left[ e^{-\Delta\tau H_0} e^{-\Delta\tau H_1} e^{\Delta\tau H_2} \right]^M} ,
    \end{aligned}
\end{equation}
where $\Delta\tau = \beta / M$ is the amount of imaginary-time between time-slices.

For the case of a lattice polaron, we have the time-discretised partition function,
\begin{equation}
    Z_M = c_1 {\Delta\tau}^{-\frac{MN}{2}} \sum_{\{r_j\}} \int \left\{ \prod_{j=1}^M \prod_{n=1}^N dx_{n,j} \right\} e^{S_{\text{ph}}} \prod_{l=1}^M I_{\Delta r_l}(2 \tau J) ,
\end{equation}
where $\Delta r_l = r_{l+1} - r_l$ is the change in the electron position across one time-slice and is an integer multiple of the lattice spacing $a$.

The discretised boson action is,
\begin{equation}
    S_{\text{ph}} = \sum_{j=1}^M \sum_{n=1}^N \left( \frac{\left( \Delta x_{n,j} \right)^2}{2 \Delta\tau} + \frac{\Delta\tau \omega^2 x^2_{n,j}}{2} + g_H \Delta\tau x_{n,j} \delta_{n,r_j}\right) .
\end{equation}
The kinetic portion of the discretised action for the fermion on a lattice is,
\begin{equation}
    I_{\Delta r_l}(z) = \frac{1}{N} \sum_{n=1}^N \cos\left( \frac{2\pi n \Delta r_l}{Na}  \right) \exp\left(-z \cos\left(\frac{2\pi n}{N}\right)\right) ,
\end{equation}
which is a discrete form of the modified Bessel function of the first-kind $I_m(z)$ \cite[\href{http://dlmf.nist.gov/10.32.E3}{(10.32.3)}]{noauthor_dlmf_nodate} where here we have $m = \Delta r_l = r_{l+1} - r_l$ and $z = 2 J \Delta\tau$. This becomes the normal modified Bessel function in the thermodynamic limit $N \to \infty$.

The bosonic integrals are Gaussian and so have closed-form expressions. By expanding the bosonic coordinates in Fourier modes,
\begin{equation}
    x_{n,j} = \frac{1}{\sqrt{M}} \sum_{q=0}^{M-1} \nu_{n,j} \exp\left( \frac{2\pi j q}{M} \right) ,
\end{equation}
we can diagonalise the bosonic action,
\begin{equation}
    S_{\text{ph}} = \sum_{n=1}^N \sum_{q=0}^{M-1} \left( \frac{\abs{\nu_{n,j}}^2}{\Delta\tau D_q^{-1}} + \frac{\Delta\tau g_H \nu_{n,q}}{\sqrt{M}} \sum_{j=1}^M \delta_{n,r_j} \exp\left(\frac{2\pi j q}{M}\right) \right) ,
\end{equation}
where 
\begin{equation}
    D_q^{-1} = 1 - \cos\left(\frac{2\pi q}{M} \right) + \frac{{\Delta\tau}^2 \omega_0^2}{2} ,
\end{equation}
is the inverse of the free-phonon Green function. Integrating over $\nu_{n,q}$ gives,
\begin{equation}
    \begin{aligned}
        Z_M &= c_2 Z^{ph}_M Z_M^{\text{el}} , \\
        Z^{ph}_M &= \left( \prod_{q=0}^{M-1} D^{1/2}_q \right)^N , \\
        Z^{\text{el}}_M &= \sum_{\{r_j\}} \left( \prod_{j=1}^M I_{\Delta r_j}(2 \Delta\tau J) \right) \exp \left( {\Delta\tau}^2 \sum_{i=1}^M \sum_{j=1}^M F(i - j) \delta_{r_i, r_j} \right) ,
    \end{aligned}
\end{equation}
where $c_2$ is just a collation of normalisation factors which will drop out of any expectation values and,
\begin{equation}
    F(l) = \frac{\Delta\tau g_H^2}{4M} \sum_{q=0}^{M-1} D_q \cos\left(\frac{2\pi q l }{M}\right) ,
\end{equation}
is the memory function that fully encodes the electron-lattice interaction over all imaginary times.

Now the kinetic portion of the action is 
\begin{equation}
    K_M[r_j] = \sum_{j=1}^M \ln\left\{I_{\Delta r_j}(2J\Delta\tau)\right\}.
\end{equation}
This path integral has no closed-form solution, even for a free electron. Regardless, we could choose the kinetic part of the trial action to be quadratic still, with an effective mass $m_e$, 
\begin{equation}
    K_{M,\text{trial}}[r_j] = \frac{m_e}{2 a^2} \sum_{j=1}^M \left(\frac{\Delta
    r_j}{\Delta\tau}\right)^2 \Delta \tau .
\end{equation}
The quasi-particle mass term $m_e$ is a variational parameter that minimises the difference between the trial and model actions. However, as the electron positions are discrete (restricted to the lattice positions), the trial path integral would be a discrete Gaussian sum, not a Gaussian integral:
\begin{equation}
    Z_{M,\text{trial}} \sim \sum_{\{r_j\}} \exp\left(K_{M,\text{trial}}[r_j]\right).
\end{equation}
This sum has no closed-form solution and would have to be evaluated numerically, defeating my goal of using the Feynman variational method to
leverage computationally efficient analytic solutions, and directly apply the FHIP theory. 

Therefore, I will take the continuum limit since we aim to use the Feynman variational method. First, I go to the continuous-time limit of the Holstein path integral.

\subsection{Continuous-Time Path Integral}
\label{subsec:3-2-2}

To obtain the continuous-time limit of the partition function, we first
explicitly remove the summation over $N$ lattice sites from the kinetic action. Doing so, we find that we have a product of these lattice-site summations for each time slice,
\begin{equation}
    Z^{\text{el}}_M = \frac{1}{(2N)^M} \sum_{\left\{r_j\right\}} \sum_{\left\{n_j\right\}} \\
    \exp\left\{i \sum_{j=1}^M \frac{2 \pi n_j}{N a} \Delta r_j \right. 
    - \left. 2 \Delta\tau J \sum_{j=1}^M \cos(\frac{2 \pi n_j}{N})\right\} ,
\end{equation}
where $n_j$ now depends on $j$. I have used the fact that the kinetic action is even with respect to $n$ and that $\cos(x) = (\exp(ix) + \exp(-ix))/2$ to change the limits of the $n_j$ summations from $\{n_j\ |\ n_j \in [1, N]\}$ to $\{n_j\ |\ n_j \in [-N, N],\ n_j \neq 0 \}$. Note that the summations over $n_j$ exclude $n_j = 0$. 

We now have a summation of all possible paths a particle can take on the
lattice within $M$ time-steps. From the kinetic action, we can see that $2\pi n_j / N a \equiv \Delta k_j$ plays the role of a discrete lattice-momenta multiplying the changes in the electron position $\Delta r_j$. 

Our electronic partition function is in the form of a discrete phase-space path integral. Therefore, in the continuous-time limit $M \to \infty$ and $\Delta \tau \to 0$, the partition function may be written as 
\begin{equation}
    Z = \mathcal{C} Z_{\text{ph}} \sum_{r(\tau)} \sum_{k(\tau)} \exp{S_{\text{eff}}} ,
\end{equation}
where $\mathcal{C}$ is the accumulation of normalisation factors. The positions $r(\tau)$ and lattice-momenta $k(\tau)$ are still restricted to $N$ discrete values. The effective action is 
\begin{equation}
    \begin{aligned}
        S_{\text{eff}} &= K[k(\tau), r(\tau)] - V_{\text{eff}}[r(\tau)] , \\
        K &= -\frac{2J}{\hbar} \int_0^{\hbar\beta} d\tau\ \cos{\left(a k(\tau)\right)} + i \int_0^{\hbar\beta} d\tau\ k(\tau) \Dot{r}(\tau) , \\
        V_{\text{eff}} &= \frac{g_H^2}{4 \hbar} \int_0^{\hbar\beta} \int_0^{\hbar\beta} d\tau d\tau' D_{\omega_0}(\tau - \tau') \delta_{r(\tau), r(\tau')} ,
    \end{aligned}
\end{equation}
where the summations over $j$ have become imaginary-time integrals and 
\begin{equation}
    \begin{aligned}
        \lim_{\Delta\tau \to 0} \left\{\Delta r_j\right\} \rightarrow\ & 
            r(\tau) , \\
        \lim_{\Delta\tau \to 0} \left\{\Delta r_j / \Delta \tau\right\} \rightarrow\ & 
            d r(\tau) / d\tau \equiv \Dot{r}(\tau), \\
        \lim_{\Delta\tau \to 0} \left\{\Delta k_j \right\} \rightarrow\
            & k(\tau) \equiv 2\pi n(\tau) / N a . 
    \end{aligned}
\end{equation}
Here $D_{\omega_0}(\tau)$ is the thermal (imaginary-time) phonon Green's function and is
\begin{equation} \label{eqn:phonongf}
    D_{\omega_0}(\tau) = \coth(\frac{\hbar\beta\omega_0}{2}) \cosh(\omega_0 \tau) - \sinh(\omega_0\tau).
\end{equation}
So far, we have derived the discrete-time Holstein path integral by starting with the Holstein Hamiltonian in a mixed representation, including phonon, electron-phonon coupling, and electronic components. We used the Trotter-Suzuki expression to discretise the quantum statistical partition function into a form suitable for path integral methods. We then expressed the partition function in terms of Bosonic and Fermionic actions, applied Gaussian integration to the Bosonic coordinates, and incorporated a variational trial action for the kinetic part of the path integral. Finally, we transitioned towards the continuum limit by reformulating the partition function in terms of continuous imaginary-time integrals, leading to an effective action that encapsulates the interactions and dynamics of the electron-phonon system.

\subsection{Thermodynamic and Continuum Limits}

In the thermodynamic limit $N\to\infty$, the summation over all discrete $k$ paths becomes continuous and can be identified with the electron quasi-momentum $k(\tau)$. The discrete sums over $k$ become continuous functional (path) integrals over quasi-momentum paths confined to the first Brillouin Zone.
\begin{equation}
    Z = \mathcal{C} Z_B \int_{r \in a \vb{Z}} \mathcal{D}r(\tau) \int_{k \in [-\frac{\pi}{a},\frac{\pi}{a}]}
    \mathcal{D} k(\tau)\ e^{S_{\text{eff}}} .
\end{equation}
The electron position is still restricted to the lattice, represented above by the set of integers, $\vb{Z}$, multiplied by the lattice constant, $a$. 

In the continuum limit $a \to 0$, the electron position can take any real value $r \in \vb{R}$, and the momentum is unbounded, $k \in (-\infty, \infty)$. Here, I assume that $a \ll 1$ so that these conditions on the electron position and momentum are approximately satisfied.

This partition function is the standard representation of the phase-space path integral, 
\begin{equation}
    Z = \mathcal{C} \int \mathcal{D}r(\tau) \int \mathcal{D}k(\tau) \exp\left\{i \int_0^{\hbar\beta} d\tau\ k(\tau) \Dot{r}(\tau) - \frac{1}{\hbar}\int_0^{\hbar\beta} d\tau\ H_{\text{eff}}[r(\tau), k(\tau)] \right\} ,
\end{equation}
where the Hamiltonian is the tight-binding Hamiltonian with an additional non-local effective interaction term,
\begin{equation}
    H_{\text{eff}} \approx 2J \cos(a k(\tau)) + g_H^2 \int_0^{\hbar\beta} d\tau'\
    D_{\omega_0} (\tau - \tau') \delta_{r(\tau),r(\tau')} .
\end{equation}
We could have started with this phase-space path integral, substituted in the Holstein Hamiltonian and performed the path integration over the lattice coordinates to arrive at the same result. Therefore, we may substitute a higher-dimensional tight-binding Hamiltonian to generalise to higher dimensions. 
The effective interaction term will be similar but with a generalised Kronecker-Delta dependent on vector positions $\Vec{r}(\tau)$.

\subsection{The Effective Mass (Parabolic-Band) Approximation}

We still face difficulty in applying the variational method. The presence of the cosine in the kinetic action (from the tight-binding band structure) renders the overall action non-convex, even in imaginary time, so Jensen's inequality does not hold. 

To make progress, we assume that $a k \ll 1$ and thereby make a parabolic (effective-mass) approximation. Since we have made the continuum approximation, this is a reasonable assumption. From the cosine form of the tight-binding band structure, and using the small-angle approximation $\cos(\theta) = 1-\theta^2$, 
\begin{equation}
    \cos(a k(\tau)) \approx 1 - \frac{a^2 [k(\tau)]^2}{2} ,
\end{equation}
In one dimension, the kinetic action is approximated by,
\begin{equation}
    K = 2 J \hbar \beta - \frac{1}{2 m_b} \int_0^{\hbar\beta} d\tau \left[k(\tau)\right]^2 + i \int_0^{\hbar\beta} d\tau k(\tau) \Dot{r}(\tau) ,
\end{equation}
where the band-mass is $m_b = \hbar^2 / 2 J a^2$. By making this approximation, the functional integral over $k(\tau)$ is the same Gaussian form as for a free particle and can be solved exactly. 

Overall, I get a parabolic-band Holstein effective action in $n$-dimensions,
\begin{equation}
    S_{\text{eff}}^{(H)} = \frac{m^{(H)}_b}{2} \int_0^{\hbar\beta} d\tau\ \vb{\Dot{r}}^2 - \frac{g_H^2}{\hbar} \int_0^{\hbar\beta} \int_0^{\hbar\beta} d\tau d\tau' D_{\omega_0}(\tau - \tau') \delta_{\vb{r}(\tau), \vb{r}(\tau')} .
\end{equation}
I have excluded the $2nJ\hbar\beta$ term since this is just the band-minimum energy absorbed into the normalisation factor $\mathcal{C}$ for the partition function. 

I reiterate my approximations to derive this action, which I will use in my numeric results. First, I went to the thermodynamic limit, which means I do not expect this model to capture any finite-size effects. Second, I make the continuum approximation so that the electron paths can be assumed to be continuous and the electron quasi-momenta unbounded. Third, I approximate the tight-binding (cosine) band with a parabolic band. 

This final effective-mass approximation means there is a missing contribution when I integrate across reciprocal space; I expect it to
break down entirely near the Brillouin-Zone boundaries. Nonetheless, the model has a genuinely short-range electron-phonon coupling. Upon integrating the phonons, the short-range electron-phonon coupling presents a non-local point-like interaction of the electron with itself through imaginary time, which is only non-zero when the electron crosses its prior path.  

Since the effective model has a Kronecker-delta-like interaction, the \emph{phonon} momentum is (correctly) bounded to remain within the first Brillouin zone. We can see this from the integral representation of the Kronecker delta, 
\begin{equation}
    \delta _{r, r'} = \frac{a}{2\pi} \int_{-\pi/a}^{\pi/a} dq\ e^{i q (r - r')} .
\end{equation}
Therefore, regarding the phonons, the effective model includes a correct description of the lattice. 

We can generalise the Kronecker-delta to arbitrary dimensions $n$ in Cartesian coordinates 
\begin{equation}
    \delta_{\vb{r}, \vb{r'}} = \frac{V_n}{(2\pi)^n} \int_{-\pi/a}^{\pi/a} d\vb{q}\ e^{i \vb{q} \cdot (\vb{r} - \vb{r'})} ,
\end{equation}
where, for example, for a cubic unit cell, $V_3 = a^3$. Alternatively, the methodology used by Feynman for the Fr\"ohlich Hamiltonian uses a Spherical coordinate representation of the Kronecker-delta in $n$-dimensions:
\begin{equation}
    \delta_{\vb{r}, \vb{r'}} = \frac{V_n \abs{S^{n-1}}}{(2\pi)^n} \int_0^{\Lambda_n} dq\ q^{n-1} e^{i q (r - r')} ,
\end{equation}
where $\Lambda_n$ is some momentum cutoff, I assume that the system has rotational invariance so that the angular components of $\vb{q} \cdot \vb{r}$ can be integrated over to give $\abs{S^{n-1}} = 2\pi^{n/2} / \Gamma(n/2)$ the surface-``area'' of an $n$-dimensional sphere where $\Gamma(x)$ is the Gamma function.

From this work on the Holstein model, we now have everything we need to
establish the machinery for a general variational method for polarons---one
that we can apply to an arbitrary electron-phonon Hamiltonian.

\section{The General Parabolic Polaron}
\label{sec:chap-third-fourth}

The Hamiltonian for a general polaron model~\cite{alexandrov_advances_2010} can be written in second-quantisation form and momentum-basis as,
\begin{empheq}{General Polaron Hamiltonian}
\begin{equation}
    \begin{aligned}
        H &= \sum_{\vb{k}} \epsilon_{\vb{k}} c^\dagger_{\vb{k}} c_{\vb{k}} + \sum_{\vb{q}} \hbar \omega_{\vb{q}} b^\dagger_{\vb{q}} b_{\vb{q}} + \sum_{\vb{k},\vb{q}} M_{\vb{k}, \vb{q}} c^\dagger_{\vb{k}+\vb{q}} c_{\vb{k}} (b^\dagger_{\vb{-q}} + b_{\vb{q}})
    \end{aligned}
\end{equation}
\end{empheq}
where $\epsilon_{\vb{k}}$ is the electron band energy for momentum $\vb{k}$, $c^\dagger_{\vb{k}}$ and $c_{\vb{k}}$ are the electron creation and annihilation operators for an electron with momentum $\vb{k}$, $\omega_{\vb{q}}$ is the phonon frequency for momentum $\vb{q}$, $b^\dagger_{\vb{q}}$ and $b_{\vb{q}}$ are the phonon creation and annihilation operators for a phonon with momentum $\vb{q}$, $M_{\vb{k}, \vb{q}}$ is the electron-phonon coupling matrix which describes the strength of the interaction.

For a model describing a parabolic band electron linearly coupled to harmonic phonons, the path integral over the phonon operators is Gaussian and can be evaluated analytically. The resultant electron action describes a temporally non-local self-interaction acting on the electron,
\begin{equation} \label{eqn:eph-action}
    \begin{aligned}
        S_{\text{pol}}[\vb{r}(\tau)] &= \frac{m_b}{2} \int_0^{\hbar\beta} d\tau\ \Dot{\vb{r}}^2(\tau) - \frac{1}{\hbar} \int_0^{\hbar\beta} d\tau \int_0^{\hbar\beta} d\tau'\ D_{\omega_0}(\tau - \tau') \Phi\left[\vb{r}(\tau), \vb{r}(\tau')\right] ,
    \end{aligned}
\end{equation}
where $ D_{\omega_0}(\tau)$ is the imaginary-time thermal phonon propagator and self-interaction functional is,
\begin{equation}
    \Phi\left[\vb{r}(\tau), \vb{r}(\tau')\right] = \sum_{\vb{q}} \abs{M_{\vb{q}}}^2 \rho_{\vb{q}} \left[\vb{r}(\tau)\right] \rho_{\vb{-q}}\left[\vb{r}(\tau')\right].
\end{equation}
Here $\rho_{\vb{q}}[\vb{r}(\tau)] = e^{i \vb{q} \cdot \vb{r}(\tau)}$ is the density for the electron derived from corresponding first-quantisation density operator and $M_{\vb{q}}$ is a general electron-phonon coupling matrix element. The polaron self-interaction functional is where the specific electron-phonon coupling presents itself in this machinery. 

For the Fr\"ohlich model, the self-interaction functional is,
\begin{equation}
    \begin{aligned}
        \Phi^{(F)}\left[\vb{r}(\tau), \vb{r}(\tau')\right] &= \sum_{\vb{q}} \frac{g_{F}^2}{V q^{n-1}} e^{i \vb{q} \cdot \left(\vb{r}(\tau) - \vb{r}(\tau') \right)} , \\
        &= g_{F}^2 \int \frac{d^n q}{(2\pi)^n} \frac{e^{i\vb{q}\cdot\left(\vb{r}(\tau) - \vb{r}(\tau')\right)}}{q^{n-1}} , \\
        &= \frac{g^2_F \abs{S^{n-1}}}{(2\pi)^{n}} \frac{1}{\abs{\vb{r}(\tau) - \vb{r}(\tau')}} ,
    \end{aligned}
\end{equation}
where $\abs{S^{n-1}} = 2\pi^{n/2}/\Gamma(n/2)$ is the hypervolume of the unit $(n-1)$-sphere and the phonon momentum is unbounded, $0 \leq q < \infty$. The Fr\"ohlich model makes the continuum approximation of the lattice, $\lim_{V \to \infty} V^{-1}\sum_{\vb{q}} \sim \int d^nq / (2\pi)^n$, where $V$ is the $n$-dimensional crystal volume.

For the Holstein model in $n$ isotropic dimensions ($n$-dimensional hypercube) self-interaction functional is 
\begin{equation}
    \begin{aligned}
        \Phi_H\left[\vb{r}(\tau), \vb{r}(\tau')\right] &= g_H^2 \sum_{\vb{q}} e^{i \vb{q} \cdot \left(\vb{r}(\tau) - \vb{r}(\tau') \right)} ,\\
        &= g_{H}^2 a^n \int \frac{d^n q}{(2\pi)^n} e^{i\vb{q}\cdot\left(\vb{r}(\tau) - \vb{r}(\tau')\right)}, \\
        &= g_H^2 \delta^n_{\vb{r}(\tau)\vb{r}(\tau')} ,
    \end{aligned}
\end{equation}
where $\delta^n_{ij}$ is the $n$-dimensional Kronecker Delta functional and $-\pi/a
\leq q \leq \pi/a$. This could be adapted to any Brillouin-Zone geometry. We could also use Spherical-Coordinates as in the Fr\"ohlich model where instead $0 \leq q \leq \Lambda_n$ with $\Lambda_n$ is a momentum cutoff given by the radius of an n-ball with volume $\frac{(2\pi)^n}{V}$:
\begin{equation}
    \Lambda_n = 2\sqrt{\pi} \left(V \Gamma\left(\frac{n}{2} + 1\right)\right)^{1/n}.
\end{equation}

\subsection{General Phonon-Band Polarons}

We can now develop the variational path integral method for this generalised polaron action and specialise to a specific case by using an explicit expression for the electron-phonon coupling in the self-interaction functional as we have above for the Fr\"ohlich and Holstein models. I will assume that we are only working with one parabolic-band electron so that the self-interaction functional contains depends on the electron position only through the term $e^{i \vb{q} \cdot (\vb{r}(\tau) - \vb{r}(\tau'))}$. If this is true, we can use Feynman's derivation of the variational path integral method.

I would also like to discuss the potential for direct numeric evaluation of the momentum integrals. The momentum integral can be evaluated numerically, either by substituting an explicit form for the electron-phonon matrix $\abs{M_{\vb{q}}}^2$ and phonon dispersion $\omega_{\vb{q}}$ and then using a numerical integration algorithm like Gauss-Kronod. Many closed-forms for the electron-phonon matrix $\abs{M_{\vb{q}}}^2$ and phonon dispersion $\omega_{\vb{q}}$ are known, such as for acoustic phonons, Bogoliubov-Fr\"ohlich polaron, impurities etc. We could use such an analytic expression in our integrals, which may then admit closed-form solutions or be evaluated numerically. 

For real materials, we could instead use electronic structure methods to evaluate $\abs{M_{\vb{q}}}^2$ and $\omega_{\vb{q}}$ on a standard reciprocal space grid. These would then enter the variational method as arrays evaluated at the electron/hole band-extremum (e.g. the gamma-point $\vb{k} = \vb{0}$). Our $q$-integrands above would become tensor products concatenated over all $q$-points.

\subsection{Effective-mass anisotropy}
\label{subsec:3-2-1}

In the degenerate anisotropic uni-axial case, I propose to na\"ively incorporate the anisotropy into the Feynman approach (which is one-dimensional due to the underlying isotropy of the Fr\"ohlich Hamiltonian) by treating the two directions independently with effective masses $m_\perp$ and $m_\parallel$.

I then use the variational principle separately in each direction to find the variational parameters $v_{\perp/\parallel}$ and $w_{\perp/\parallel}$ that give the lowest upper-bound to the ground-state energy $E_{\perp/\parallel}$ for each direction. 

The variational parameters can then be used to obtain the effective polaron masses $M_{\perp}$ and $M_{\parallel}$ using Eq. (\ref{eqn:mass_feynman}) and polaron radii $R_{\perp}$ and $R_{\parallel}$ using Eq. (\ref{eqn:pol_size_schultz}). 

To make comparisons with the isotropic case, I define an effective ground-state energy by taking the arithmetic mean of the uniaxial components of the ground-state energy,
\begin{equation}
    E = \frac{2 E_\perp + E_\parallel}{3}.
\end{equation}
Similarly, I define an effective radius of the anisotropic polaron by finding the radius of a sphere with the same volume as the ellipsoidal anisotropic polaron. This means taking the geometric mean of the uni-axial components of the polaron radius, 
\begin{equation}
    R = \left(R^2_{\perp} R_{\parallel}\right)^{1/3}.
\end{equation}
The two averaging methods are justified as they are the only ones that give a ground-state energy and polaron radius consistent with those evaluated by the original model,~\cite{feynman_slow_1955}, when applied to an isotropic material.

\section{Many Polaron Theory}

So far, I have derived path integrals using the eigenvalues of the position and momenta operators of the Polaron Hamiltonian. However, there is another construction based on the eigenvalues of the Fock-Space annihilation operators resulting in the so-called \emph{Coherent-State Path Integrals} (CSPI)~\cite{altland_condensed_2010}.

Intuitively, CSPI allows us to construct a path integral directly from the second-quantised form of the Hamiltonian rather than the first-quantised form. Statistically, this corresponds to constructing the \emph{grand-canonical} ensemble partition function in the path integral formalism compared to the canonical ensemble partition function we have been using previously. Thus, using CSPIs allows us to construct path integrals for systems with varying particle numbers and may allow us to generalise the variational principle to non-quadratic kinetic action.

Excluding electronic spin degrees-of-freedom and the potential for multiple phonon branches, we may write the general polaron Hamiltonian:
\begin{equation}
    H_{\text{polaron}} = \sum_{\vb{k}n} \epsilon_{\vb{k}n} c^\dagger_{\vb{k}n} c_{\vb{k}n} + \sum_{\vb{q}j} \hbar \omega_{\vb{q}j} b^\dagger_{\vb{q}j} b_{\vb{q}j} + \sum_{\vb{k}mn, \vb{q}j} M_{\vb{k}mn,\vb{q}j} c^\dagger_{\vb{k}+\vb{q}m}c_{\vb{k}n} \left( b^\dagger_{-\vb{q}j} + b_{\vb{q}j} \right)
\end{equation}
% Propose a trial Hamiltonian where we uncorrelated the electron and phonon momenta:

% \begin{equation}
%     H_{\text{trial}} = \sum_{\vb{k}} \epsilon_{\vb{k}} c^\dagger_{\vb{k}} c_{\vb{k}} + \sum_{\vb{q}} \hbar \Omega_{\vb{q}} b^\dagger_{\vb{q}} b_{\vb{q}} + \sum_{\vb{k}, \vb{q}} M_{\vb{k},\vb{q}} c^\dagger_{\vb{k}}c_{\vb{k}} \left( b^\dagger_{-\vb{q}} + b_{\vb{q}} \right)
% \end{equation}

% where $\Omega_{\vb{q}}$ and $M_{\vb{k}, \vb{q}}$ will be variational parameters.

The coherent state path integral for the polaron Hamiltonian $H_{\text{polaron}}$:
\begin{equation}
    Z_{\text{polaron}} = \int D[\Bar{c}, c] \int D[\Bar{b},b]\ e^{-S_{\text{el}}[\Bar{c}, c] - S_{\text{ph}}[\Bar{b},b] - S_{\text{el-ph}}[\Bar{c},c,\Bar{b},b]}
\end{equation}
with the free-electron action:
\begin{equation}
        S_{\text{el}}\left[\Bar{c}, c\right] =  \sum_{\vb{k}n} \int_0^{\hbar\beta} d\tau\ \Bar{c}_{\vb{k}n}(\tau) \left[\partial_{\tau} - \epsilon_{\vb{k}n} - \mu \right] c_{\vb{k}n}(\tau)
\end{equation}
the free-phonon action:
\begin{equation}
        S_{\text{ph}}\left[\Bar{b}, b\right] =  \sum_{\vb{q}j} \int_0^{\hbar\beta} d\tau\ \Bar{b}_{\vb{q}j}(\tau) \left[\partial_{\tau} - \omega_{\vb{q}j} \right] b_{\vb{q}j}(\tau)
\end{equation}
and the electron-phonon interaction action:
\begin{equation}
    \begin{aligned}
        S_{\text{el-ph}}\left[\Bar{c}, c, \Bar{b}, b\right] &=  \sum_{\vb{k}mn,\vb{q}j} M_{\vb{k}mn, \vb{q}j} \int_0^{\hbar\beta} d\tau\ \Bar{c}_{\vb{k}+\vb{q}m}(\tau) c_{\vb{k}n}(\tau) b_{\vb{q}j}(\tau) \\
        &+ \sum_{\vb{k}mn,\vb{q}j} M_{\vb{k}mn, \vb{q}j} \int_0^{\hbar\beta} d\tau\ \Bar{b}_{-\vb{q}j}(\tau) \Bar{c}_{\vb{k}+\vb{q}m}(\tau) c_{\vb{k}n}(\tau)
    \end{aligned}
\end{equation}
The phonon contribution to the action is quadratic and may be integrated to give:
\begin{equation}
    Z^0_{\text{ph}} \exp{\frac{1}{2} \sum_{\vb{q}, \vb{k}, \vb{k}'} M_{\vb{k}, \vb{q}} M^*_{\vb{k}', \vb{q}} \int_0^{\hbar\beta} \int_0^{\hbar\beta} d\tau d\tau'\ \Bar{c}_{\vb{k}}(\tau) c_{\vb{k} - \vb{q}} (\tau) D^0_{\omega_{\vb{q}}}(\tau - \tau') \Bar{c}_{\vb{k}'-\vb{q}}(\tau') c_{\vb{k}'}(\tau')}
\end{equation}
where $D^0_{\omega_{\vb{q}}}(\tau)$ is the bare phonon kernel for a harmonic oscillator and $Z^0_{\text{ph}} = \det
\left[D^0\right]^{-1/2}$ is the non-interacting phonon partition function.

The coherent state path integral for the polaron Hamiltonian $H_{\text{polaron}}$:

\begin{equation}
    Z_{\text{polaron}} = \int D\Bar{c} Dc \exp{-S_{\text{polaron}}\left[\Bar{c}(\tau), c(\tau)\right]}
\end{equation}

with action:

\begin{equation}
    \begin{aligned}
        S_{\text{polaron}}\left[\Bar{c}(\tau), c(\tau)\right] =  \sum_{\vb{k}} &\int_0^{\hbar\beta} d\tau\ \Bar{c}_{\vb{k}}(\tau) \left[\partial_{\tau} - \epsilon_{\vb{k}} - \mu \right] c_{\vb{k}}(\tau) \\
        - \frac{1}{2} \sum_{\vb{k}, \vb{q}} \abs{M_{\vb{k}, \vb{q}}}^2 \int_0^{\hbar\beta} &\int_0^{\hbar\beta} d\tau d\tau'\ \Bar{c}_{\vb{k}}(\tau) c_{\vb{k}+\vb{q}}(\tau) D^0_{\omega_{\vb{q}}}(\tau - \tau') \Bar{c}_{\vb{k}+\vb{q}}(\tau') c_{\vb{k}}(\tau') 
    \end{aligned}
\end{equation}