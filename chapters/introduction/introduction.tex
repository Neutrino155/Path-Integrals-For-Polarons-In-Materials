\clearpage{}

\pagestyle{body}

\chapter{Introduction}
\label{chap:first}

\thesisepisrcyear{\justifying{``The underlying physical laws necessary for the mathematical theory of a large part of physics and the whole of chemistry are thus completely known, and the difficulty is only that the exact application of these laws leads to equations much too complicated to be soluble. It therefore becomes desirable that approximate practical methods of applying quantum mechanics should be developed, which can lead to an explanation of the main features of complex atomic systems without too much computation.''}}{Paul A. M. Dirac}{Quantum mechanics of many-electron systems}{1929}

\chapterintrobox{This is the introduction paragraph.}

\section{The Polaron}
\label{sec:chap-first-first}

\section{Polaron Mobility}
\label{sec:chap-first-second}

\section{Relevance: Semiconductor Materials}
\label{sec:chap-first-third}

\lettrine{S}emiconductor materials have enormous technical application in ranging from photovoltaic cells and light-emitting diodes to field-effect transistors and solid-state lasers. A key phenomenological quantity that influences the performance of a semiconductor is the charge carrier mobility. The required mobility depends on the application; for example, a solid-state laser require higher mobility than a photovoltaic cell. As a phenomenological quantity, mobility is not a direct ground state property. Instead, mobility arises from competitive dynamic processes within a material and can on occasion be strongly temperature-dependent. Most mobility theories tend to use semi-classical approximations for modelling the mobility; using an effective scattering time such as in Drude like models, or neglecting electron-phonon interactions as in the adiabatic Born-Oppenheimer approximation. Otherwise, it is usually assumed that the electron-phonon coupling is either weak or strong, where approximate limits are taken. However, some semiconductors, such as halide perovskites, possess large dielectric electron-phonon interactions, leading to the formation of large polarons. These materials typically exhibit intermediate electron-phonon coupling strengths. Alternatively, one can use the path integral formulation of quantum mechanics to derive an inherently quantum theory of mobility (\cite{feynman_slow_1955, feynman_mobility_1962}). Though based on a linear-response theory, this is non-perturbative and can provide a temperature-dependent model of mobility to all orders in the electron-phonon coupling strength.